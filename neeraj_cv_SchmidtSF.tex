%______________________________________________________________________________________________________________________
% @brief    LaTeX2e Resume for Kamil K Wojcicki
\documentclass[line]{resume}
\usepackage{hyperref}
%\usepackage[paperwidth=25cm,paperheight=22cm,left=1cm,top=1cm]{geometry}
\usepackage[paperwidth=21cm,paperheight=30cm,left=2cm,top=2cm]{geometry}
\renewcommand{\familydefault}{\sfdefault}
\fontfamily{garamond}
%______________________________________________________________________________________________________________________
\begin{document}
\small{
   \name{[CV] \Large Neeraj Kumar Sharma}
\begin{resume}
    \setcounter{page}{1}
\pagenumbering{arabic} 
%     \vspace{-1.5mm}
%      PhD Student
     \vspace{-4.5mm}
%__________________________________________________________________________________________________________________
    % Contact Information
%    \section{\mysidestyle Contact\\Information}
    \vspace{1mm}
    BrainHub Fellow            					\hfill e-mail: neerajww@gmail.com \\
    Learning and Extraction of Acoustic Patterns Laboratory     \hfill website: \url{https://neerajww.github.io}\\
%    Electrical Engineering 	                \\
    Indian Institute of Science (IISc), Bangalore, India              \\\vspace{-4.5mm}%
%__________________________________________________________________________________________________________________
%     \section{\mysidestyle Date of Birth}
%     11 June 1987,  Age: 30
 %     \vspace{-2.5mm}
%  \section{\mysidestyle Goal} To work for an internship in a dynamic environment exploring an exciting area in signal processing
% applications and DSP coding. 
%__________________________________________________________________________________________________________________
%     % Current Research Work
     \vspace{-2.5mm}
     \section{\mysidestyle Research Outline}
     \vspace{1mm}
     \textbf{Keywords:} Speech and audio signal analysis (sampling, analysis, modification, and reconstruction),
     Time-frequency analysis, and psychoacoustics.\\
     \textbf{Thesis title:} Information-rich Sparse sampling of Time-varying Signals\\
     \newline\noindent\textbf{Abstract} Sound signals such as speech, and birdsongs are composed of time-varying oscillations.
     Depending on the underlying physical principle used in the generation of the sound, the time-varyingness in the oscillaitons
     can range between very high to very low.
     Interestingly, our auditory system analyzes this time-varyingness to trigger our perception, and cognition.
     And it does this with a performance which is unparalled when compared to any sound analyzing system.
     Motivated to improve the design of sound analysis systems, the thesis contributions are two fold.
     Firstly, it analyzes time-varyingness in speech signals without assuming any quasi-stationarity (a striking contrast to
     conventional approaches). It is found that the proposed analysis (and algorithms) lend \textit{analysis-modification-synthesis} of speech
     in a manner which preserves the naturalness. The results show that non-paramteric processing of time-varying attributes in speech
     may be key in encoding perceived information.
     Secondly, the thesis proposes an event-trigerred sampling (ETS) paradigm to analyze time-varyingness in non-stationary signals, such as speech.
     In ETS scheme, a sample is drawn from the continuous-time signal whenever the a preset event occurs.
     The analyzed preset events are higher-order zero-crossings of the signal, and the focus is on capturing the time-varyingness
     in signals using these events.
     The idea behind proposing using ETS is to approximate the event-locked firing observed experimentally in synapses between inner-hair cells and auditory nerve fibers.
     ETS results in sub-Nyquist rate samples which are non-equispaced in time. Algorithms are devised to analyze these samples.
     It is found that processing samples captured via ETS has potential benefits, and performance in estimation of time-varying parameters can
     outperform that obtained with conventional uniform Nyquist-rate sampling.
     The areas explored in the research contributing to the thesis include the following.
     \vspace{-3.5mm}
     \begin{itemize}
      \item sampling theory, signal representations, transforms, and sparse signal processing.
      \item speech acoustics, auditory processing, and psychoacoustics
      \item feature extraction, and classification
     \end{itemize}
   \vspace{-1.5mm}
%__________________________________________________________________________________________________________________
    % Education 
    \section{\mysidestyle Education}
    \vspace{1mm}
    \textbf{PhD Scholar} \hfill \textbf{ August 2009 - June 2017 (thesis submitted)}\vspace{-3mm}\\\vspace{-1mm}\\%
    Indian Institute of Science, Bangalore, India \vspace{0mm}\\%
    Advisor:  Prof.~Dr.~Thippur V. Sreenivas, Dept. ECE, IISc\vspace{-6mm}\\

    \textbf{Bachelor of Technology} \hfill \textbf{ August 2005-2009}\vspace{-3mm}\\\vspace{-1mm}\\%
    Instrumentation and Electronics Engineering\\
    College of Engineering and Technology, Bhubaneswar, India \vspace{0mm}\\%
    CGPA: 9.21/10 \vspace{-6mm}\\

    \textbf{12th Board Schooling} \hfill \textbf{ 2003-2005}\vspace{-3mm}\\\vspace{-1mm}\\%
    DAV Public School, Unit-8, Bhubaneswar, India \vspace{0mm}\\%
    Percentage: 86\%, Central Board of Secondary Education\vspace{-6mm}\\
    
    \textbf{10th Board Schooling} \hfill \textbf{ 2002-2003}\vspace{-3mm}\\\vspace{-1mm}\\%
    DAV Public School, Unit-8, Bhubaneswar, India \vspace{0mm}\\%
    Percentage: 88\%, Central Board of Secondary Education
    \vspace{-1.5mm}
%__________________________________________________________________________________________________________________
    % Education 
    \section{\mysidestyle Additional Research Exposure}
    \vspace{1mm}
    \textbf{Visiting Faculty Staff} \hfill \textbf{ July -- Present, 2017}\vspace{-3mm}\\\vspace{-1mm}\\%
    Speech Learning and Perception Laboratory\\
    Carnegie Mellon University, Pittsburgh, USA \vspace{0mm}\\%
    Mentor: Prof.~Dr.~Lori H.~Holt\vspace{-6mm}\\

    \textbf{BrainHub Fellow} \hfill \textbf{ March -- Present, 2017}\vspace{-3mm}\\\vspace{-1mm}\\%
    Learning and Extraction of Acoustic Patterns Laboratory, IISc\vspace{0mm}\\%
    Mentor: Dr.~Sriram Ganapathy\vspace{-6mm}\\

    \textbf{Project Staff} \hfill \textbf{ April -- Oct, 2016}\vspace{-3mm}\\\vspace{-1mm}\\%
    Large Scale Audio Analytics, Speech and Audio Group, IISc\\
    Mentor: Prof.~Dr.~Thippur~V.~Sreenivas\vspace{-6mm}\\

    \textbf{Visiting Researcher} \hfill \textbf{ April -- July, 2014}\vspace{-3mm}\\\vspace{-1mm}\\%
    Audition Lab, Ecole Normale Superieure, Paris, France \vspace{0mm}\\%
    Mentor: Dr.~Daniel Pressnitzer (ENS, Paris), and Dr.~Laurent Daudet (ESPCI, Paris)\vspace{-6mm}\\

\vspace{-1.5mm}
%__________________________________________________________________________________________________________________
 % TAs
    \section{\mysidestyle Teaching Assistantship}
    \vspace{1mm}
%    \begin{list2}
       \textbf{Time Frequency Analysis (E9-213)} Jan-May 2012 in IISc. The resposibility involved mentoring on assignments.\vspace{1mm}\\%
       \textbf{Signal Quantization and Compression (E9-221)} Aug-Dec 2011 in IISc. The responsibility involved mentoring on
       assignments and course project.\vspace{1mm}
%    \end{list2}
    \vspace{-1.5mm}
%__________________________________________________________________________________________________________________
 % PMs
    \section{\mysidestyle Research Mentorship}
    \vspace{1mm}
%    \begin{list2}
       \textbf{May-July 2017} Siddharth Mittal, from IIT Kanpur, on automatic detection of speaker change instants in speech signals.\vspace{1mm}\\%
       \textbf{Jan-March 2016} Shreepad Potadar, from NIT Surathkal, on time-scale modification of speech and audio signals.\vspace{1mm}\\%
       \textbf{Jan-Feb 2016} Anil Sharma, from IIIT Delhi, on scream classification using auditory cortical features.\vspace{1mm}\\%
       \textbf{May-July, 2015} Santhosh Gandreti, from IIT Bhubaneswar, on sound event classification using spectro-temporal analysis.\vspace{1mm}\\%
       \textbf{April-May, 2015} Pawan Kumar Rukmangada, from BMS College Bangalore, on database creation for speaker verification over telephone IVRS.\vspace{1mm}\\%
       \textbf{May-July, 2013} Amrutha Nadarajan, from NIT Trichy, on zero-crossings analysis of speech.
%    \end{list2}
    \vspace{-1.5mm}
%__________________________________________________________________________________________________________________    
 % Honours and Awards
    \section{\mysidestyle Awards}
    \vspace{1mm}
    BrainHub Carnegie Mellon University - IISc Fellowship \hfill \textbf{2017} \vspace{1mm}\\%
    IEEE-Eta Kappa Nu (IEEE-HKN) Memberbership, the honor society of IEEE\hfill \textbf{2017} \vspace{1mm}\\%
%    Student Travel Grant Mechanics of Hearing (MoH) to attend MoH Workshop held in Athens, 2014. \vspace{1mm}\\%
%    SPIE Officer Travel Grant Student Chapter to attend Leadership Workshop held in Brussels, 2014. \vspace{1mm}\\%
    IEEE MV Chauhan Paper Contest First Prize Winner \hfill \textbf{2013}					   \vspace{1mm}\\%
    IEEE-IISc Student Branch Best Volunteer Award \hfill \textbf{2012-13}					   \vspace{1mm}\\%
%    IEEE Signal Processing Society Student Travel Grant to attend ICASSP held in Kyoto, 2012. \vspace{1mm}\\%
    PhD Scholarship, Ministry of Human Resource and Development, Govt. of India \hfill \textbf{2009-2015} \vspace{1mm}\\%
    Finalist in Motorola Scholar Program, Innovative Project Design \hfill \textbf{2009}                          \vspace{1mm}\\%
    Undergraduate Merit Scholarship, Govt. of Odisha \hfill \textbf{2005-2009}                             \vspace{1mm}\\%
    Third Prize in National Level Paper Contest on Renewable Energy, Dhenkanal \hfill \textbf{2008}               \vspace{1mm}\\%   
    Winning entry in the UMO Boycott Bad Design Contest \hfill \textbf{2008}                                      \vspace{1mm}\\%
    Selected in State Level Chemistry Olympiad \hfill \textbf{2004}                                               
    \vspace{-1.5mm}
%__________________________________________________________________________________________________________________    
 % Honours and Awards
    \section{\mysidestyle Grants}
    \vspace{1mm}
    Student Travel Grant Mechanics of Hearing (MoH) to attend MoH Workshop held in Athens \hfill \textbf{2014} \vspace{1mm}\\%
    SPIE Officer Travel Grant Student Chapter to attend Leadership Workshop held in Brussels \hfill \textbf{2014} \vspace{1mm}\\%
    IEEE Signal Processing Society Student Travel Grant to attend ICASSP held in Kyoto \hfill \textbf{2012}
    \vspace{-1.5mm}
%__________________________________________________________________________________________________________________
% Publications
    \section{\mysidestyle Publications}
    \vspace{5mm}
    \begin{list2}
       \item Neeraj Sharma, Shreepad  Potadar, Srikanth Raj Chetupalli and  T.~V.~Sreenivas, ``Mel-Scale Sub-band Modelling for Perceptually
       Improved Time-Scale  Modification of Speech and Audio  Signals'' (under review),
	\textsl{in Proc. 23rd National Conference on Communications (NCC)}, March 2017, Madras, India.       
      \item Neeraj Sharma, and T.~V.~Sreenivas, ``Event-triggered sampling using signal extrema for instantaneous amplitude and instantaneous frequency estimation'',
	\textsl{Elsevier Signal Processing}, 2015.
      \item Neeraj Sharma, ``Time-instant sampling based encoding of time-varying acoustic spectrum'',
	\textsl{AIP Conf. Proc. of Intl. Conf. on Mechanics of Hearing}, 2015.
      \item Neeraj Sharma, Sai Gunaranjan Pelluri, and T.~V.~Sreenivas, ``Moving acoustic source parameter estimation using a single microphone
      and signal extrema samples'',
	\textsl{IEEE Intl. Conf. on Acoustics, Speech, and Signal Processing (ICASSP)}, April 2015, Brisbane, Australia.
      \item Neeraj K. Sharma and T.~V.~Sreenivas, ``Event-trigerred sampling and reconstruction of sparse trigonometric polynomials'',
	\textsl{IEEE Intl. Conf. on Signal Processing and Communications (SPCOM)}, July 2014, Bangaore, India.
      \item Neeraj Sharma and T.~V.~Sreenivas, ``Sparse signal reconstruction based on non-uniform signal dependent samples'',
	\textsl{IEEE Intl. Conf. on Acoustics, Speech, and Signal Processing (ICASSP)}, March 2012, Kyoto, Japan.
     \end{list2}
    \vspace{-1mm}
     Manuscripts in preparation:
     \begin{list2}
       \item Neeraj Sharma, and T.~V.~Sreenivas, ``Time-varying quasi-harmonic modeling of speech''.
       \item Neeraj Sharma, and T.~V.~Sreenivas, ``Multi-component time-varying sinusoidal processing using Higher-Order Zero-Crossings''.
       \item Neeraj Sharma, and T.~V.~Sreenivas, ``Speech and audio processing using event-triggered sampling of Higher-order Zero-Crossings''.
       \end{list2}
     \vspace{-1.5mm}
%__________________________________________________________________________________________________________________
% Technical Schools
    \vspace{-2.5mm}
    \section{\mysidestyle Talks Delivered} 
    \vspace{5mm}
    \begin{list2}
    \item Implications From Audition: Informative Instants for Non-Stationary Signal Analysis, in Five Minutes PhD Thesis Presentation at
    Workshop on Speech Source Modelling \& its Applications, July 2016, Gandhinagar.
    \item Sound processing , invited talk in Annual Techfest of National Institute of Design, 2015, Mysore.
    \item Throwing Light into the Tunnel: auditory models and perception, invited talk in Winter School on Speech and Audio Processing (WiSSAP) 2015, held in Gandhinagar, India.
    \item Detect and Sample: Questioning uniform Nyquist-rate sampling, invited talk in IEEE Day Celebrations, Oct 2013, at IISc.
    \item Sound signal analysis: Some knowns and unknowns, invited talk in SIAM-IISc Chapter, May 2015, at IISc.
    \item Understanding Signals, in Knowledge Outreach Programme, at Govt. SKSJTI College, Bangalore, Sept. 2013.
    \item Meaning of Signal Analysis, in Outreach Programme, at College of Engineering and Technology (CET), Bhubaneswar, Dec. 2012.
    \end{list2}
%__________________________________________________________________________________________________________________
% Courses Crdeited
    \vspace{-2.5mm}
    \section{\mysidestyle Graduate Courses} 
    \vspace{2.5mm}
    Time-Frequency Analysis, Random Processes, Pattern Recognition and Neural Networks, Adaptive Signal Processing, Matrix Theory,
    Digital Signal Compression, Non-linear Signal Processing, Stochastic Models for Speech Recognition, and Digital Image Processing.
%__________________________________________________________________________________________________________________ 
% Computer Skills
    \vspace{-3.5mm}
    \section{\mysidestyle Scientific Techniques} 
    \vspace{2.5mm}
     Coding in MATLAB, Python, HTML, Java Script, Shell \vspace{1mm}\\
     Operating system usage UNIX \vspace{1mm}\\
     Report documentation with \LaTeXe
%__________________________________________________________________________________________________________________
% Hobbies
    \vspace{-3.5mm}
    \section{\mysidestyle Administration Experience}
    \vspace{2.5mm}
    Co-maintained the Dept. of ECE website in 2014-15 \vspace{1mm}\\
    Executive Committee Member of IEEE-IISc Student Branch
%__________________________________________________________________________________________________________________
% Hobbies
    \vspace{-3.5mm}
    \section{\mysidestyle Academic Service}
    \vspace{2.5mm}
    Reviewer for Elsevier Signal Processing journal, ISCA conference Interspeech in 2015 and 2017,
    IEEE Signal Processing Letters, IEEE Connect conference 2014, IEEE Signal Processing and Communication (SPCOM) Conference 2012, and Sadhana journal.\vspace{1mm}\\
    Contributed as Organization Committee member in the following events. 
    \begin{list2}
     \item Winter School in Speech and Audio Processing, 2012. This is an annual international school on speech and audio processing.
     \item Sparse Signal Processing Course, 2012, hosted by IEEE Bangalore Chapter.
     \item Annual Electrical Sciences Divisional Symposium, 2014, IISc. An annual symposium for graduating students of IISc to talk
     about their findings and collaborate with industries.
     \item Open Day, 2012, 2013, 2014. An annual one day event hosted at IISc to make public familiar with activities at the institute.
    \end{list2}
%__________________________________________________________________________________________________________________
%     \section{\mysidestyle Referees} 
%     {\sl Available on request.}
% %______________________________________________________________________________________________________________________
\newpage
% \vspace{-4.5mm}
\section{\mysidestyle Referees} 
\vspace{-5.5mm}
\begin{tabular}{@{}p{6cm}p{6cm}}
% \textbf{Professor Thippur V. Sreenivas}       &  \textbf{Dr Stephen So}                   \\
% Professor, Dept. ECE                          &  Associate Lecturer                       \\
% Indian Institute of Science             &  Griffith University                      \\
% Banglore, India                         &  Gold Coast, Queensland, Australia        \\
% % phone: \textsl{available on request}    &  phone: \textsl{available on request}     \\
% % e-mail: \textsl{available on request}   &  e-mail: \textsl{available on request}    \\
\vspace{6mm}
\textbf{Thippur V. Sreenivas}                         \\
Professor, Dept. ECE                                            \\
Indian Institute of Science                                     \\
Banglore, India                                                 \\
% phone: \textsl{available on request}    &  phone: \textsl{available on request}     \\
e-mail: \textsl{tvsree@iisc.ac.in}  \\


\vspace{6mm}
\textbf{Sriram Ganapathy}                         \\
Assistant Professor, Dept. EE                                   \\
Indian Institute of Science                                     \\
Banglore, India                                                 \\
% phone: \textsl{available on request}    &  phone: \textsl{available on request}     \\
e-mail: \textsl{sriramg@iisc.ac.in}    \\

\vspace{6mm}
\textbf{Lori Holt}                         \\
Professor, Dept. Psychology                                   \\
Carnegie Mellon University (CMU)                                \\
Pittsburgh, US                                                 \\
% phone: \textsl{available on request}    &  phone: \textsl{available on request}     \\
e-mail: \textsl{loriholt@cmu.edu}    \\


\vspace{6mm}
\textbf{Daniel Pressnitzer}                         \\
Director, Audition Lab,                                            \\
Ecole Normale Superieure (ENS)                                     \\
Paris, France                                                 \\
% phone: \textsl{available on request}    &  phone: \textsl{available on request}     \\
e-mail: \textsl{Daniel.Pressnitzer@ens.fr}    \\
\vspace{6mm}
\textbf{Chandra Sekhar Seelamantula}                         \\
Associate Professor, Dept. EE                                   \\
Indian Institute of Science                                     \\
Banglore, India                                                 \\
% phone: \textsl{available on request}    &  phone: \textsl{available on request}     \\
e-mail: \textsl{chandra.sekhar@ee.iisc.ernet.in}    \\


\vspace{6mm}
\textbf{Shihab A. Shamma}                         \\
Professor                                   \\
University of Maryland                                \\
Baltimore, US                                                 \\
% phone: \textsl{available on request}    &  phone: \textsl{available on request}     \\
e-mail: \textsl{sas@isr.umd.edu}    \\

% \vspace{6mm}
% \textbf{K.V.S. Hari}                         \\
% Professor, Dept. ECE                                   \\
% Indian Institute of Science                                     \\
% Banglore, India                                                 \\
% phone: \textsl{available on request}    &  phone: \textsl{available on request}     \\
% e-mail: \textsl{available on request}   &  e-mail: \textsl{available on request}    \\
\end{tabular}

%______________________________________________________________________________________________________________________
\end{resume}

\end{document}


%______________________________________________________________________________________________________________________
% EOF

