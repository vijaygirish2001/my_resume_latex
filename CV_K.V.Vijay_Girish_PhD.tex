\documentclass[10pt]{article}
\usepackage{charter}
\usepackage{fullpage}
\usepackage[colorlinks=false]{hyperref}
\usepackage{currvita}

% Better for lists with 1-2 items and short descriptions
\newenvironment{sublist}{%
	\begin{list}{}{%
		\setlength{\itemsep}{0em}\setlength{\parsep}{0em}%
		\setlength{\topsep}{0em}\setlength{\parskip}{0em}%
	}%
}%
{ \end{list} }

% Better for lists with more than 2 i cotems and/or long descriptions
\newenvironment{subbulletlist}{%
	\begin{list}{\labelitemii}{%
		\setlength{\topsep}{\itemsep}\setlength{\parskip}{\parsep}%
	}%
}%
{ \end{list} }

\pagestyle{empty}

\begin{document}

\newlength{\oldcvlabelwidth}
\newlength{\oldcvlabelsep}

\begin{cv}{{\large K Venkata Vijay Girish}\\
{ \normalsize  PhD (Completed), MILE Lab, Department of Electrical Engineering, \\Indian Institute of Science, Bangalore- 560012, INDIA
\\
Email: {\mdseries \href{mailto:vijay.girish@gmail.com}
	{vijay.girish@gmail.com}}
\hfill Phone: {\mdseries +91 9480515318} \hfill \\
Web: {\mdseries \href{https://sites.google.com/site/kvvijaygirish/}
	{https://sites.google.com/site/kvvijaygirish/}}}
}

\setlength{\oldcvlabelwidth}{\cvlabelwidth}
\setlength{\oldcvlabelsep}{\cvlabelsep}

\setlength{\cvlabelwidth}{1em}


\setlength{\cvlabelwidth}{0em}
\setlength{\cvlabelsep}{\labelsep}
\begin{cvlist}{Date of Birth}\item
23-10-1985
\end{cvlist}


\begin{cvlist}{Research Interests}
\item 
I am  interested in designing innovative mathematical and machine learning models to analyze, process and extract useful information from various signals and data. I am inclined towards mathematical, analytical and  out of the box thinking to solve any problem. 
 My broad areas of interest are:
\begin{itemize}\itemsep=0.25em
	\item Machine Listening: Audio Classification and Analysis pertaining to Speaker, Background Noise and Language Classification, Source Separation, Audio Signal Analysis, Audio Segmentation, Speech Enhancement and Multilingual Speech Recognition
	
 \item Machine Learning, Artificial Intelligence, Deep learning, Data Analytics and Signal Processing
 
% 
\item Sparse Signal Processing, Dictionary Learning and Adaptation, Image and Video  Processing

\end{itemize}
\end{cvlist}


\begin{cvlist}{Education}
	\item \emph{PhD, Systems and Signal Processing, August 2010- April 2017, September 2017 (Thesis submitted, Thesis Defense)}\\
	Department of Electrical
Engineering, Indian Institute of Science, Bangalore, GPA: 6.0/8.0
	\begin{subbulletlist}
		\item \emph{Research Advisors: Prof. A. G. Ramakrishnan, Department of Electrical Engineering, Indian Institute of Science and Dr. T. V. Ananthapadmanabha, Voice and Speech Systems, Bangalore
}
		
	\end{subbulletlist}
	\item \emph{B.Tech., Electrical and Electronics Engineering}, August 2004- May 2008\\
	National Institute of Technology Karnataka, Surathkal, GPA: 7.75/10.00
	\item \emph{Class- 10+2, CBSE}, May 2003 \\
	Kendriya Vidyalaya No.-2, Kharagpur, Percentage: 82.0
	\item \emph{Class- 10, CBSE}, May 2001 \\
	Kendriya Vidyalaya No.-2, Kharagpur, Percentage: 75.8
\end{cvlist}



\begin{cvlist}{PhD Thesis Summary}
\item
\textit{Title:} Speech and noise analysis using sparse representation and acoustic phonetics knowledge.
  
  This thesis  addresses  different  aspects of speech and noise analysis using two different approaches, namely (1) A supervised and adaptive sparse representation based approach for identifying the type of background noise and the speaker and separating the speech and background noise, and (2) An unsupervised acoustic-phonetics knowledge  based approach for detecting transitions between broad phonetic classes  and significant excitation instants called as glottal closure instants (GCIs) in a speech signal, for applications like speech segmentation, recognition and modification.  
    	 
    	
    	In the supervised sparse representation based approach, a dictionary
    	learning based noise classification algorithm is proposed using a cosine
    	similarity measure for learning atoms of the dictionary. We have used the
    	Active Set Newton Algorithm (ASNA) and supervised non-negative matrix factorization  for source recovery in the testing
    	phase. Based on the objective measure of signal to distortion ratio (SDR),
     we get frame-wise noise  classification accuracy of 97.8\% for fifteen different noise sources taken from NOISEX database.
    	For speaker classification on clean speech,  using  high energy subsets of test frames and dictionary atoms, sum of weights measure on concatenated dictionaries give good accuracy. 
    	 	We have then dealt with  noisy speech signals assuming a single speaker speaking in a noisy environment. The speaker and/or  the noise   belonging to unknown sources is handled by adaptation. Adaptive noise dictionary gives an
    	 	    	improvement of about 18\% in speaker classification accuracy and  4 dB in SDR over an out-of-set dictionary, after
    	 	    	enhancement of noisy speech at an SNR of 0 dB.
    	 	    	  	  We have also addressed the classification of speakers and subsequent separation of speakers in  overlapped speech, obtaining a  mean speaker classification accuracy of 84\% for the  speaker 1 to speaker 2 ratio (S1S2R) of 0 dB.
    	 	
    
    	
    	
    	In the unsupervised acoustic-phonetics knowledge based approach, we detect
    	transitions between broad phonetic classes in a speech signal which has
    	applications such as landmark detection and segmentation. A rule-based
    	approach using relative thresholds learnt from a small development set is
    	devised to detect transitions of silence to non-silence, sonorant to
    	non-sonorant and vice-versa. This approach does not require significant
    	training data for determining the parameters of the proposed approach. When
    	tested on the entire TIMIT database for clean speech, 93.6\% of the detected
    	transitions are within a tolerance of 20 ms from the hand labeled
    	boundaries. The proposed method is also tested on the test set of the TIMIT
    	database for robustness with respect to white, babble and Schroeder noise,
    	and about 90\% of the detected transitions are within a tolerance of 20 ms
    	at a SNR of 5 dB. 
    	
    	We have also proposed subband analysis of linear prediction
    	residual (LPR) to estimate the GCIs from voiced speech segments. The GCI detection performance of the proposed algorithm is quantified using the following measures: identification rate (IDR), miss rate (MR), false alarm rate (FAR), standard deviation of error (SDE) and accuracy to 0.25 ms. It is evaluated using 6 different
    	databases and compared with 3 state-of-the-art LPR based methods. The
    	proposed method is comparable to the best of the LPR based techniques for
    	clean and noisy speech.
    	

\end{cvlist}

\begin{cvlist}{Industry experience}\item
\begin{itemize}\itemsep=0.25em
\item \emph{Senior Technical Leader, Huawei Technologies India,} April 2017- present

\begin{itemize}
\item Analysis of universities and preparing proposals for collaboration and setting up Innovation lab
\item Analysis and review of machine learning services on various cloud platforms
\item Brainstorming and  proposed patents on noise and audio analysis for cloud platform
\item  AI/ML resource analysis and gave technical talks on machine learning and cloud services
\item  Created proposals for getting new project on audio detection/analysis, speaker classification and voice assistant on Edge and Cloud platform
\item Submitted research papers on distributed audio classification  and anomaly prediction
\end{itemize}

	\item \emph{Senior Engineer, Relay and Integrated
	Solutions, Larsen and Toubro Limited, Mumbai,} August 2008- July 2010 
\begin{itemize}
 \item
Development of new product: Intelligent Motor Protection Relay, MCOMP
\item
Indepth testing, troubleshooting and analysis of new product
\item
New product documentation and management
\item
Development of Data Concentrator Systems

\end{itemize}




\item
\emph{Industrial Training:
Inplant Practical Training at the Bharat Heavy
Electricals Limited, Electronics Division, Bangalore,}
May 2006-June 2006


	
\end{itemize}



\end{cvlist}

\begin{cvlist}{Research Publications}
\item \textbf{Conference publications}
\item Published:
	\begin{itemize}\itemsep=0.25em
	\item Sayan Ghosh, K V Vijay Girish, T.V. Sreenivas, \textit{Relationship between Indian Languages Using Long Distance Bigram Language Models}, Proc. International Conference on Natural Language Processing (ICON
2011), Dec 16-19, 2011, Chennai, India, pp. 104-113


\item
Vikram Ramesh Lakkavalli, K V Vijay Girish, A G Ramakrishnan, \textit{Sub-band Envelope Approach to Obtain Instants of Significant Excitation in Speech}, Proc. National Conference on Communications (NCC 2012), Feb 3-5, 2012, Kharagpur, India, pp. 19

\item Vikram R L, K V Vijay Girish, Harshavardhan S, A G Ramakrishnan, T V Ananthapadmanabha, 
\textit{Subband Analysis of Linear Prediction Residual for the Estimation of Glottal Closure Instants}, Proc. IEEE International Conference on Acoustics, Speech and Signal Processing (ICASSP 2014), May 4-9, 2014, Florence, Italy

\item	K V Vijay Girish, A G Ramakrishnan and T V Ananthapadmanabha, \textit{Hierarchical classification of speaker and background noise and estimation of SNR using sparse representation}, Interspeech 2016, September 8-12, 2016, San Francisco

\item  K V Vijay Girish, Veena Vijai and A G Ramakrishnan,  \textit{Relationship between spoken Indian languages by clustering of long distance bigram features of speech},  INDICON 2016, IISc Bangalore

\item  K V Vijay Girish, T V Ananthapadmanabha and A G Ramakrishnan,  \textit{Cosine similarity based dictionary learning and source recovery for classification of diverse audio sources}, INDICON 2016, IISc Bangalore
\item K V Vijay Girish and A G Ramakrishnan, \textit{Enhancement of noisy Tamil speech for improved quality of perception for the hearing impaired}, 16th Tamil Internet Conference 2017, Toronto
\item K V Vijay Girish, A G Ramakrishnan and Neeraj Kumar, \textit{A system for distributed audio classification using sparse representation over cloud for IOT}, accepted in COMSNETS 2018, Bangalore
	\end{itemize}

\item \textbf{Journal publications}
\item Under review: 
\begin{itemize}
\item T V Ananthapadmanabha, K V Vijay Girish and A G Ramakrishnan,  \textit{Relative occurrences and difference of extrema for detection of transitions between broad phonetic classes}, submitted to Sadhana

	\end{itemize}
	


\end{cvlist}

\begin{cvlist}{Technical Reports}
\item
\begin{itemize}
\item T V Ananthapadmanabha, K V Vijay Girish, A G Ramakrishnan, \textit{Detection of transitions between broad phonetic classes in a speech signal}, 	arXiv:1411.0370 [cs.SD]
%\item K V Vijay Girish, T V Ananthapadmanabha, A G Ramakrishnan, \textit{A dictionary learning and source recovery based approach to classify diverse audio sources}, arXiv, Submitted on 27 Oct 2015

\item  K V Vijay Girish, A G Ramakrishnan and T V Ananthapadmanabha,  \textit{Adaptive dictionary based approach for background noise and speaker classification and subsequent source separation}, 	arXiv:1609.09764 [cs.SD]
\end{itemize}
\end{cvlist}

\begin{cvlist}{Technical Skills}
\item
\begin{itemize}\itemsep=0.25em
	\item Relevant Subjects:\\
	\textbf{During B.Tech}  \textit{August 2004 - May 2008:}\\
 \textit{Computer Programming, Introduction to Algorithms and Data Structures, Numerical Methods, Digital Signal Processing, Digital System Design,  Microprocessors,
Computer Organization and Architecture}\\
\textbf{During PhD} \textit{August 2010- current:}\\
\textit{Matrix Theory,
Linear and Non Linear Optimization, Probability Theory,  Convex Optimization,
Advanced Digital Signal Processing,
Pattern Recognition and Neural Networks, Machine Learning, Data Mining, Compressive Sensing and Sparse Signal Processing, Time Frequency Analysis, Speech Information Processing, Automatic Speech Recognition Algorithms, Digital Image Processing}



	\item Programming languages known:
 \textit{C, Python, Matlab, VHDL,  Latex}

\item Softwares used: \textit{Maxwell, PSpice, Xilinx, Modelsim, Matlab, Simulink and Praat}
\item Operating system: \textit{Worked on Windows XP and Linux}
%\item Others: \textit{Microsoft Office}



	\end{itemize}

\end{cvlist}




\begin{cvlist}{Project Mentorship}
\item
\begin{itemize}


\item Veena Vijai, from Birla Institute of Technology and Science, Pilani - K. K. Birla Goa Campus  on Relationship between spoken Indian languages by clustering of long distance bigram features of speech, May-July, 2016

\item B Shubashree, from SSN College of Engineering, Chennai on Unsupervised background noise change identification using dictionary learning, June- August, 2016
\end{itemize}

\end{cvlist}

\begin{cvlist}{Project experience}
	\item
\begin{itemize}
	\item \textbf{Project Associate} DRDO project on Speaker and background change detection from August 2016 to April 2017
	\begin{itemize}
		\item Understanding and delivering the project requirements
		\item Regular meeting with DRDO scientists for understanding their requirements and  presenting work progress
		\item Visited DRDO CAIR Lab for testing algorithms on the data provided by DRDO 
	\end{itemize}
\end{itemize}	
\end{cvlist}


\begin{cvlist}{Teaching Experience}

\item 
 \textbf{Linear and Nonlinear Optimization} offered  by Prof. Muthuvel Arigovindan at  Indian Institute of Science, Bangalore during 
 August-December, 2013: Responsibilities include conducting tutorial classes  
\item
 \textbf{Speech Information Processing}  offered  by Prof. A G Ramakrishnan at  Indian Institute of Science, Bangalore during 
  January-May, 2014: Responsibilities include   preparing assignments and projects
  \item
   \textbf{Matrix Theory}  offered  by Prof. A G Ramakrishnan at  Indian Institute of Science, Bangalore during 
    August-December, 2014: Responsibilities include   preparing assignments, clearing doubts, evaluation and grading of students.
      

\end{cvlist}

\begin{cvlist}{Academic Achievements}
\item
\begin{itemize}\itemsep=0.25em
	\item
Got through AIEEE 2004 with an All India Rank of 4525 and State Rank (West Bengal) of 53

\item

 Secured an All India Rank of 5439 in IIT-JEE 2004

\item
Got through GATE 2010 (Electrical Engineering) with 98.8 percentile

\item
Has been among top 3 students of the school consistently

\item
Distinctive performance in 2nd National Science Olympiad in 2000


	\end{itemize}

\end{cvlist}



\begin{cvlist}{Talks and Poster Presentations}
\item
\begin{itemize}
 \item Assisted Prof. A. G. Ramakrishnan in conducting a tutorial on Insights into Signal Processing,Transforms and Linear Algebra at International Conference on Biomedical Engineering, 2011 held at MIT Manipal, during 8-9 December, 2011


\item 

Gave a talk on my accepted paper in  National Conference on Communications,  held at IIT Kharagpur, during 3-5 February, 2012
\item Presented poster on my accepted paper in IEEE International Conference on Acoustics, Speech and Signal Processing (ICASSP 2014) at Florence, Italy  during  4-9 May, 2014

\item 
Presented poster on my accepted paper in INTERSPEECH-2016 at Hyatt Regency, San Francisco, USA during 8-12 September, 2016

\item Gave a talk and poster presentation on ``Analysis of audio intercepts: Can we identify and locate the speaker? " at EECS Research Students Symposium - 2016,  held at IISc Bangalore during 28-29 April, 2016
\item Presented poster on ``Adaptive and supervised sparse representation based approach for noisy speech analysis"  in IISconnect: Industry Interaction Day at IISc  Bangalore on 3rd October, 2016 
\item Gave an IEEE talk on ``Throwing light on sound" in NIT Goa and BITS Goa, on 12th February, 2017
\end{itemize}
\end{cvlist}

%
%\begin{cvlist}{Workshops and  Conferences}
%\item 
%\begin{itemize}\itemsep=0.25em
%\item Attended  Centenary Conference, Electrical Engineering held at IISc Bangalore, during 14-17 December, 2011
%\item Attended a Workshop on Image and Speech Processing, WISP-2011 held at IIIT Hyderabad,  on 17th December, 2011
%
%\item Attended WiSSAP-2012 on Compuatational Auditory Scene Analysis (CASA), held at IISc Bangalore, during 6-9 January, 2012
%
%\item Attended  One Day Workshop on Image Processing Using LabVIEW,  held at IISc Bangalore, on 25 February, 2012
%\item Attended One-day Workshop on Speech Processing and Applications, held at CMRIT Bangalore, on 21 June, 2012
%\item Attended Winter School and Conference on Computational Aspects of Neural Engineering, held at IISc Bangalore, during  12 - 21 December, 2012
%\item Attended WiSSAP-2013 on Statistical Parametric Speech Synthesis, held at IIT Madras, during 22-25 February, 2013
%
%\item Attended a 12-week course on the Science of Scientific Writing during August  to November, 2013 by Dr. Karthik Ramaswamy, held at IISc, Bangalore
%
%
%\item Attended WiSSAP-2015 on Production-Perception Based New Models of Speech Analysis held at Dhirubhai Ambani 
%Institute of Information and Communication Technology (DA-IICT), Gandhinagar, India  during 4-7 January, 2015 
%\item Attended Learning  sparse representations for Signal Processing held at IISc Bangalore, India during  20 - 22 February, 2015
%\item Attended WiSSAP-2016 on Speech Prosody held at SSN College of Engineering, Chennai, India during 8-11 January, 2016
%
%
%\item Attended CHIME Workshop, The 4th International Workshop on 
%Speech Processing in Everyday Environments at Google office, San  Francisco, USA on 13 September, 2016
%
%
%	\end{itemize}
%\end{cvlist}
%
%


\begin{cvlist}{Hobbies and Extra-curricular activities}
\item
\begin{itemize}\itemsep=0.25em
\item Participated in Google Code Jam in the past five years
	\item
Bagged 2nd prize in Foxhunt in the TechFest Engineer 2006
conducted by NITK Surathkal
\item
Secured 5th position in Simplicity, an International
Online Matlab Programming contest conducted during
Engineer 2008, a Technical Festival conducted by NITK
Surathkal

\item Bagged 3rd prize in Science Slam in Pravega Sci-Tech 2014 conducted by Swissnex at IISc Bangalore

\item Volunteer in the organization team of IISc Electrical Sciences Divisional Symposium, 2013
\item


Active Member of  Spicmacay, Voice club, IEEE and Management Forum
 at NITK Surathkal
 \item Active member of Mess committee at IISc Bangalore from July 2013 - May 2015
\item

Playing badminton, guitar,  photography,  running, swimming, biking, triathlon,  programming, technology, reading, writing blogs


	\end{itemize}

\end{cvlist}


\begin{cvlist}{Languages Known}
\item
\begin{itemize}\itemsep=0.25em
	\item English, Hindi and Telugu
	\end{itemize}

\end{cvlist}





\begin{cvlist}{References}
\item  \textbf{Prof. A G Ramakrishnan}\\
Chairman, Department of Electrical Engineering,\\
Indian Institute of Science Bangalore, India.\\
E-mail: ramkiag@ee.iisc.ernet.in

\item \textbf{Dr. T V Ananthapadmanabha}\\
Founder and Chief innovator,\\
 Voice and Speech Systems, Bangalore, India\\
E-mail:  tva.blr@gmail.com

\item \textbf{Prof. T V Sreenivas}\\
Professor, Department of Electrical Communication Engineering, \\
Indian Institute of Science Bangalore, India.\\
E-mail: tvsreenivas@gmail.com
\end{cvlist}



\setlength{\cvlabelwidth}{\oldcvlabelwidth}
\setlength{\cvlabelsep}{\oldcvlabelsep}

\end{cv}
\end{document}

