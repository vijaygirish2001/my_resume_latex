\documentclass[10pt]{article}
\usepackage{charter}
\usepackage{fullpage}
\usepackage[colorlinks=false]{hyperref}
\usepackage{currvita}

% Better for lists with 1-2 items and short descriptions
\newenvironment{sublist}{%
	\begin{list}{}{%
		\setlength{\itemsep}{0em}\setlength{\parsep}{0em}%
		\setlength{\topsep}{0em}\setlength{\parskip}{0em}%
	}%
}%
{ \end{list} }

% Better for lists with more than 2 i cotems and/or long descriptions
\newenvironment{subbulletlist}{%
	\begin{list}{\labelitemii}{%
		\setlength{\topsep}{\itemsep}\setlength{\parskip}{\parsep}%
	}%
}%
{ \end{list} }

\pagestyle{empty}

\begin{document}

\newlength{\oldcvlabelwidth}
\newlength{\oldcvlabelsep}

\begin{cv}{{\large K Venkata Vijay Girish}\\
{ \normalsize  PhD Student (Thesis submitted), MILE Lab, Department of Electrical Engineering, \\Indian Institute of Science, Bangalore- 560012, INDIA
\\
Email: {\mdseries \href{mailto:vijay.girish@gmail.com}
	{vijay.girish@gmail.com}}
\hfill Phone: {\mdseries +91 9480515318} \hfill \\
Web: {\mdseries \href{https://sites.google.com/site/kvvijaygirish/}
	{https://sites.google.com/site/kvvijaygirish/}}}
}

\setlength{\oldcvlabelwidth}{\cvlabelwidth}
\setlength{\oldcvlabelsep}{\cvlabelsep}

\setlength{\cvlabelwidth}{1em}


\setlength{\cvlabelwidth}{0em}
\setlength{\cvlabelsep}{\labelsep}
\begin{cvlist}{Date of Birth}\item
23-10-1985
\end{cvlist}

\begin{cvlist}{Present position}\item
\emph{Senior Technical Leader, Huawei Technologies India, April 2017- present}\\
Worked on:
\begin{itemize}
\item Machine learning for cloud computing and big data: anomaly prediction
\item Preparing proposals for university collaboration
\item Analysis of Machine Learning as a Service on various cloud platforms

\end{itemize}
\end{cvlist}
\begin{cvlist}{Research Interests}\item
\begin{itemize}\itemsep=0.25em
 \item Machine Learning, Data Analytics and Audio Signal Processing
 
% 
\item Sparse Signal Processing, Dictionary Learning and Adaptation, Image and Video  Processing
\item Machine Listening: Audio Source Classification pertaining to Speaker, Background Noise and Language Classification, Source Separation, Audio Signal Analysis and Detection of Transitions

\end{itemize}
\end{cvlist}


\begin{cvlist}{Education}
	\item \emph{PhD, Systems and Signal Processing, August 2010- April 2017}\\
	Department of Electrical
Engineering, Indian Institute of Science, Bangalore, GPA: 6.0/8.0
	\begin{subbulletlist}
		\item \emph{Research Advisors: Prof. A. G. Ramakrishnan, Department of Electrical Engineering, Indian Institute of Science and Dr. T. V. Ananthapadmanabha, Voice and Speech Systems, Bangalore
}
		
	\end{subbulletlist}
	\item \emph{B.Tech., Electrical and Electronics Engineering}, August 2004- May 2008\\
	National Institute of Technology Karnataka, Surathkal, GPA: 7.75/10.00
	\item \emph{Class- 10+2, CBSE}, May 2003 \\
	Kendriya Vidyalaya No.-2, Kharagpur, Percentage: 82.0
	\item \emph{Class- 10, CBSE}, May 2001 \\
	Kendriya Vidyalaya No.-2, Kharagpur, Percentage: 75.8
\end{cvlist}


\begin{cvlist}{Research Publications}
\item \textbf{Conference publications}
\item Published:
	\begin{itemize}\itemsep=0.25em
	\item Sayan Ghosh, K V Vijay Girish, T.V. Sreenivas, \textit{Relationship between Indian Languages Using Long Distance Bigram Language Models}, Proc. International Conference on Natural Language Processing (ICON
2011), Dec 16-19, 2011, Chennai, India, pp. 104-113


\item
Vikram Ramesh Lakkavalli, K V Vijay Girish, A G Ramakrishnan, \textit{Sub-band Envelope Approach to Obtain Instants of Significant Excitation in Speech}, Proc. National Conference on Communications (NCC 2012), Feb 3-5, 2012, Kharagpur, India, pp. 19

\item Vikram R L, K V Vijay Girish, Harshavardhan S, A G Ramakrishnan, T V Ananthapadmanabha, 
\textit{Subband Analysis of Linear Prediction Residual for the Estimation of Glottal Closure Instants}, Proc. IEEE International Conference on Acoustics, Speech and Signal Processing (ICASSP 2014), May 4-9, 2014, Florence, Italy

\item	K V Vijay Girish, A G Ramakrishnan and T V Ananthapadmanabha, \textit{Hierarchical classification of speaker and background noise and estimation of SNR using sparse representation}, Interspeech 2016, September 8-12, 2016, San Francisco

\item  K V Vijay Girish, Veena Vijai and A G Ramakrishnan,  \textit{Relationship between spoken Indian languages by clustering of long distance bigram features of speech},  INDICON 2016, IISc Bangalore

\item  K V Vijay Girish, T V Ananthapadmanabha and A G Ramakrishnan,  \textit{Cosine similarity based dictionary learning and source recovery for classification of diverse audio sources}, INDICON 2016, IISc Bangalore
	\end{itemize}

%\item \textbf{Journal publications}
%\item Submitted: 
%\begin{itemize}
%\item  K V Vijay Girish, A G Ramakrishnan and T V Ananthapadmanabha,  \textit{Adaptive dictionary based approach for background noise and speaker classification and subsequent source separation}, submitted to IEEE/ACM TASLP Special Issue on Sound Scene and Event Analysis
%
%	\end{itemize}
	


\end{cvlist}

\begin{cvlist}{Technical Reports}
\item
\begin{itemize}
\item T V Ananthapadmanabha, K V Vijay Girish, A G Ramakrishnan, \textit{Detection of transitions between broad phonetic classes in a speech signal}, 	arXiv:1411.0370 [cs.SD]
%\item K V Vijay Girish, T V Ananthapadmanabha, A G Ramakrishnan, \textit{A dictionary learning and source recovery based approach to classify diverse audio sources}, arXiv, Submitted on 27 Oct 2015

\item  K V Vijay Girish, A G Ramakrishnan and T V Ananthapadmanabha,  \textit{Adaptive dictionary based approach for background noise and speaker classification and subsequent source separation}, 	arXiv:1609.09764 [cs.SD]
\end{itemize}
\end{cvlist}

\begin{cvlist}{Technical Skills}
\item
\begin{itemize}\itemsep=0.25em
	\item Relevant Subjects:\\
	\textbf{During B.Tech}  \textit{August 2004 - May 2008:}\\
 \textit{Computer Programming, Introduction to Algorithms and Data Structures, Numerical Methods, Digital Signal Processing, Digital System Design,  Microprocessors,
Computer Organization and Architecture}\\
\textbf{During PhD} \textit{August 2010- current:}\\
\textit{Matrix Theory,
Linear and Non Linear Optimization, Probability Theory,  Convex Optimization,
Advanced Digital Signal Processing,
Pattern Recognition and Neural Networks, Machine Learning, Data Mining, Compressive Sensing and Sparse Signal Processing, Time Frequency Analysis, Speech Information Processing, Automatic Speech Recognition Algorithms, Digital Image Processing}



	\item Programming languages known:
 \textit{C, Python, Matlab, VHDL,  Latex}

\item Softwares used: \textit{Maxwell, PSpice, Xilinx, Modelsim, Matlab, Simulink and Praat}
\item Operating system: \textit{Worked on Windows XP and Linux}
%\item Others: \textit{Microsoft Office}



	\end{itemize}

\end{cvlist}


\begin{cvlist}{Academic Projects}
\item \textbf{ During PhD} \textit{August 2010- current:}
\begin{itemize}\itemsep=0.25em
	\item
	Relationship Between Indian Languages Using Bigram Language Models- Group
project under Prof. T. V. Sreenivas of Dept of Electrical Communication Engineering,
Indian Institute of Science

\item
Voiced and Unvoiced feature classification of Speech data using multiple features-
Individual course project under Prof. A. G. Ramakrishnan of Dept of Electrical
Engineering, Indian Institute of Science

\item
Implementation of Modification of Pitch contour , Duration and Energy   for change in Prosody in Tamil Speech Synthesis under Prof. A. G. Ramakrishnan of Dept of Electrical
Engineering, Indian Institute of Science
\item
Voice Modification system using change of sampling frequency and duration normalization under Prof. A. G. Ramakrishnan of Dept of Electrical
Engineering, Indian Institute of Science
\item
Implementation of Parts of Speech and Pause Rules Tagging of Tamil text in Matlab and C under Prof. A. G. Ramakrishnan of Dept of Electrical
Engineering, Indian Institute of Science
\item
Robust estimation of glottal closure instants using dynamic weighting of subband components in Speech under Prof. A. G. Ramakrishnan of Dept of Electrical
Engineering, Indian Institute of Science

\item Subband Analysis of Linear Prediction Residual for the Estimation of Glottal Closure Instants under Dr. T. V. Ananthapadmanabha of Voice and Speech Systems, Bangalore and Prof. A. G. Ramakrishnan of Dept of Electrical Engineering, Indian Institute of Science

\item Detection of transitions among broad phonetic classes in a speech signal
          using temporal features and a rule based approach under Dr. T. V. Ananthapadmanabha of Voice and Speech Systems, Bangalore and Prof. A. G. Ramakrishnan of Dept of Electrical Engineering, Indian Institute of Science

\item A dictionary learning and source recovery based approach to classify diverse
                               audio sources under Dr. T. V. Ananthapadmanabha of Voice and Speech Systems, Bangalore and Prof. A. G. Ramakrishnan of Dept of Electrical Engineering, Indian Institute of Science
                               
\item Hierarchical classification of speaker and background noise and estimation of SNR using sparse representation under Prof. A. G. Ramakrishnan of Dept of Electrical Engineering, Indian Institute of Science and  Dr. T. V. Ananthapadmanabha of Voice and Speech Systems, Bangalore  

\item Supervised  and adaptive dictionary learning using  sparse representation for noisy speech analysis and language classification under Prof. A. G. Ramakrishnan of Dept of Electrical Engineering, Indian Institute of Science and  Dr. T. V. Ananthapadmanabha of Voice and Speech Systems, Bangalore  
	\end{itemize}

\item \textbf{ During B.Tech} \textit{August 2004 - May 2008:}
\begin{itemize}\itemsep=0.25em
	\item \textit{Major Project:} Steady State Analysis of Unified Power Flow Controller using Matlab 5.3 and
Simulink 3.0- group project under Dr. K.N. Shubhanga of Dept of Electrical and
Electronics Engineering, NITK Surathkal

\item
Digital Quiz- group project under Mr. K. Manjunath
Sharma of Dept of Electrical and Electronics
Engineering, NITK Surathkal
\item

Pspice Simulation of high frequency signal only during
negative half cycle of the power supply- individual
project under Mr. A.R.Beig of Dept of Electrical and
Electronics Engineering, NITK Surathkal

\item
Design and implementation of Badminton Scoreboard (for
doubles match) using Xilinx VHDL software- individual project
under Prof. P.Vittal of Dept of Electrical and
Electronics Engineering, NITK Surathkal

\item
Graphic Equalizer implementation using Matlab 7.0
- individual project under Prof. Jora M Gonda of Dept of
Electrical and Electronics Engineering, NITK Surathkal

\item
Design of a a simplified bus model of Embedded
Intel486 SX Processor using Xilinx VHDL software- group
project under Prof. P.Vittal of Dept of Electrical
and Electronics Engineering, NITK Surathkal

\item
2-Dimensional electromagnetic field simulation for
high performance electromechanical design using Maxwell
SV software- group project under Dr.P.Duraikannu
of Dept of Electrical and Electronics Engineering,
NITK Surathkal

\item
Speed Control of DC Motor using PC Generated Pulse Width
Modulation (Hardware Project)- group project under Mrs. Vinatha U. of
Dept of Electrical and Electronics Engineering, NITK
Surathkal

\item
Implementation of Image Processing Techniques using Graphical User
Interface in Matlab 7.0.1- group project under Mrs. Vinatha U. of Dept of Electrical
and Electronics Engineering, NITK Surathkal





\end{itemize}
\end{cvlist}



\begin{cvlist}{Project Mentorship}
\item
\begin{itemize}


\item Veena Vijai, from Birla Institute of Technology and Science, Pilani - K. K. Birla Goa Campus  on Relationship between spoken Indian languages by clustering of long distance bigram features of speech, May-July, 2016

\item B Shubashree, from SSN College of Engineering, Chennai on Unsupervised background noise change identification using dictionary learning, June- August, 2016
\end{itemize}

\end{cvlist}

\begin{cvlist}{Teaching Experience}

\item 
 \textbf{Linear and Nonlinear Optimization} offered  by Prof. Muthuvel Arigovindan at  Indian Institute of Science, Bangalore during 
 August-December, 2013: Responsibilities include conducting tutorial classes  
\item
 \textbf{Speech Information Processing}  offered  by Prof. A G Ramakrishnan at  Indian Institute of Science, Bangalore during 
  January-May, 2014: Responsibilities include   preparing assignments and projects
  \item
   \textbf{Matrix Theory}  offered  by Prof. A G Ramakrishnan at  Indian Institute of Science, Bangalore during 
    August-December, 2014: Responsibilities include   preparing assignments, clearing doubts, evaluation and grading of students.
      

\end{cvlist}
\begin{cvlist}{Technical Experience}
\item
\begin{itemize}\itemsep=0.25em
	\item Work experience of 2 years from August- 2008 to July- 2010 which includes 1st year as
Graduate Engineer Trainee and 2nd year as a Senior Engineer in Relay and Integrated
Solutions, Larsen and Toubro Limited, Mumbai
\begin{itemize}
 \item
Development of new product: Intelligent Motor Protection Relay, MCOMP
\item
Indepth testing, troubleshooting and analysis of new product
\item
New product documentation and management
\item
Development of Data Concentrator Systems

\end{itemize}




\item
Industrial Training:
Inplant Practical Training at the Bharat Heavy
Electricals Limited, Electronics Division, Bangalore
(May-June 2006)


	
\end{itemize}
\end{cvlist}

\begin{cvlist}{Academic Achievements}
\item
\begin{itemize}\itemsep=0.25em
	\item
Got through AIEEE 2004 with an All India Rank of 4525 and State Rank (West Bengal) of 53

\item

 Secured an All India Rank of 5439 in IIT-JEE 2004

\item
Got through GATE 2010 (Electrical Engineering) with 98.8 percentile

\item
Has been among top 3 students of the school consistently

\item
Distinctive performance in 2nd National Science Olympiad in 2000


	\end{itemize}

\end{cvlist}



\begin{cvlist}{Talks and Poster Presentations}
\item
\begin{itemize}
 \item Assisted Prof. A. G. Ramakrishnan in conducting a tutorial on Insights into Signal Processing,Transforms and Linear Algebra at International Conference on Biomedical Engineering, 2011 held at MIT Manipal, during 8-9 December, 2011


\item 

Gave a talk on my accepted paper in  National Conference on Communications,  held at IIT Kharagpur, during 3-5 February, 2012
\item Presented poster on my accepted paper in IEEE International Conference on Acoustics, Speech and Signal Processing (ICASSP 2014) at Florence, Italy  during  4-9 May, 2014

\item 
Presented poster on my accepted paper in INTERSPEECH-2016 at Hyatt Regency, San Francisco, USA during 8-12 September, 2016

\item Gave a talk and poster presentation on ``Analysis of audio intercepts: Can we identify and locate the speaker? " at EECS Research Students Symposium - 2016,  held at IISc Bangalore during 28-29 April, 2016
\item Presented poster on ``Adaptive and supervised sparse representation based approach for noisy speech analysis"  in IISconnect: Industry Interaction Day at IISc  Bangalore on 3rd October, 2016 
\item Gave an IEEE talk on ``Throwing light on sound" in NIT Goa and BITS Goa, on 12th February, 2017
\end{itemize}
\end{cvlist}

%
%\begin{cvlist}{Workshops and  Conferences}
%\item 
%\begin{itemize}\itemsep=0.25em
%\item Attended  Centenary Conference, Electrical Engineering held at IISc Bangalore, during 14-17 December, 2011
%\item Attended a Workshop on Image and Speech Processing, WISP-2011 held at IIIT Hyderabad,  on 17th December, 2011
%
%\item Attended WiSSAP-2012 on Compuatational Auditory Scene Analysis (CASA), held at IISc Bangalore, during 6-9 January, 2012
%
%\item Attended  One Day Workshop on Image Processing Using LabVIEW,  held at IISc Bangalore, on 25 February, 2012
%\item Attended One-day Workshop on Speech Processing and Applications, held at CMRIT Bangalore, on 21 June, 2012
%\item Attended Winter School and Conference on Computational Aspects of Neural Engineering, held at IISc Bangalore, during  12 - 21 December, 2012
%\item Attended WiSSAP-2013 on Statistical Parametric Speech Synthesis, held at IIT Madras, during 22-25 February, 2013
%
%\item Attended a 12-week course on the Science of Scientific Writing during August  to November, 2013 by Dr. Karthik Ramaswamy, held at IISc, Bangalore
%
%
%\item Attended WiSSAP-2015 on Production-Perception Based New Models of Speech Analysis held at Dhirubhai Ambani 
%Institute of Information and Communication Technology (DA-IICT), Gandhinagar, India  during 4-7 January, 2015 
%\item Attended Learning  sparse representations for Signal Processing held at IISc Bangalore, India during  20 - 22 February, 2015
%\item Attended WiSSAP-2016 on Speech Prosody held at SSN College of Engineering, Chennai, India during 8-11 January, 2016
%
%
%\item Attended CHIME Workshop, The 4th International Workshop on 
%Speech Processing in Everyday Environments at Google office, San  Francisco, USA on 13 September, 2016
%
%
%	\end{itemize}
%\end{cvlist}
%
%


\begin{cvlist}{Hobbies and Extra-curricular activities}
\item
\begin{itemize}\itemsep=0.25em
\item Participated in Google Code Jam in the past five years
	\item
Bagged 2nd prize in Foxhunt in the TechFest Engineer 2006
conducted by NITK Surathkal
\item
Secured 5th position in Simplicity, an International
Online Matlab Programming contest conducted during
Engineer 2008, a Technical Festival conducted by NITK
Surathkal

\item Bagged 3rd prize in Science Slam in Pravega Sci-Tech 2014 conducted by Swissnex at IISc Bangalore

\item Volunteer in the organization team of IISc Electrical Sciences Divisional Symposium, 2013
\item


Active Member of  Spicmacay, Voice club, IEEE and Management Forum
 at NITK Surathkal
 \item Active member of Mess committee at IISc Bangalore from July 2013 - May 2015
\item

Playing badminton, guitar,  photography,  running, swimming, biking, triathlon,  programming, technology, reading, writing blogs


	\end{itemize}

\end{cvlist}


\begin{cvlist}{Languages Known}
\item
\begin{itemize}\itemsep=0.25em
	\item English, Hindi and Telugu
	\end{itemize}

\end{cvlist}




\setlength{\cvlabelwidth}{\oldcvlabelwidth}
\setlength{\cvlabelsep}{\oldcvlabelsep}

\end{cv}
\end{document}

