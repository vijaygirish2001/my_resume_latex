%______________________________________________________________________________________________________________________
% @brief    LaTeX2e Resume for Kamil K Wojcicki
\documentclass[line]{resume}
\usepackage{hyperref}
%\usepackage[paperwidth=25cm,paperheight=22cm,left=1cm,top=1cm]{geometry}
\usepackage[paperwidth=21cm,paperheight=30cm,left=2cm,top=1.5cm]{geometry}
\renewcommand{\familydefault}{\sfdefault}
\fontfamily{garamond}
%______________________________________________________________________________________________________________________
\begin{document}
	\small{
		\name{ \Large K V Vijay Girish}
		\begin{resume}
			\setcounter{page}{1}
			\pagenumbering{arabic} 
			%     \vspace{-1.5mm}
			%      PhD Student
			\vspace{-4.5mm}
			%__________________________________________________________________________________________________________________
			% Contact Information
			%    \section{\mysidestyle Contact\\Information}
			\vspace{1mm}
		 Data Scientist Tech Lead          					\hfill e-mail: vijay.girish@gmail.com \\
			Belong.co,   \hfill Website: \url{https://sites.google.com/site/kvvijaygirish/}\\
			%    Electrical Engineering 	                \\
		   Bangalore, INDIA  \hfill Linkedin: \url{https://www.linkedin.com/in/k-v-vijay-girish-b85a3714/}         \\			
		 \hspace{5cm}  Mobile no.: +91 9480515318 \hfill Github: \url{https://github.com/vijaygirish2001}
		 %\vspace{-1.5mm}%
			%__________________________________________________________________________________________________________________
			%     \section{\mysidestyle Date of Birth}
			%     11 June 1987,  Age: 30
			%     \vspace{-2.5mm}
			%  \section{\mysidestyle Goal} To work for an internship in a dynamic environment exploring an exciting area in signal processing
			% applications and DSP coding. 
			%__________________________________________________________________________________________________________________
			%     % Current Research Work
			%\vspace{-2.5mm}
			\section{\mysidestyle Research Interests}
			\vspace{1mm}
			I am  interested in designing innovative mathematical and machine learning models to analyze, process and extract  information from signals and data  captured from various domains like audio, vision and text. I enjoy analytical and emperical modelling, and  out of the box thinking to solve practical problems. 
			My broad areas of interest are:\vspace{-.1cm}
			
			 Artificial Intelligence, Machine Learning, Deep learning, Data Analytics, Signal Processing,  Sparse Representations, Dictionary Learning and Adaptation, Speech and Audio Analysis, Source Separation, Acoustic Noise and Speaker Classification,  Speech Recognition, Keyword Spotting, Image and Video  Processing
			\vspace{-.1cm}
			
			I am especially passionate about deep tech problems, mathematical models, audio and speech research, and their applications to real world problems.
      	\section{\mysidestyle PhD Thesis Summary}
      	
      	\textbf{Thesis title:} Speech and noise analysis using sparse representation and acoustic phonetics knowledge\\	
			\noindent\textbf{Abstract}  This thesis  addresses  different  aspects of speech and noise analysis using two different approaches:\vspace{-.5cm}
			\begin{enumerate}
				\item  A supervised and adaptive sparse representation based approach for identifying the type of background noise and the speaker and separating the speech and background noise, and
				\item An unsupervised acoustic-phonetics knowledge  based approach for detecting transitions between broad phonetic classes  and significant excitation instants called as glottal closure instants (GCIs) in a speech signal, for applications like speech segmentation, recognition and modification
			\end{enumerate} 
			
			
		%	In the supervised sparse representation based approach, a dictionary			learning based noise classification algorithm is proposed using a cosine			similarity measure for learning atoms of the dictionary. We have used the			Active Set Newton Algorithm (ASNA) and supervised non-negative matrix factorization  for source recovery in the testing			phase. 	For speaker classification on clean speech,  using  high energy subsets of test frames and dictionary atoms, sum of weights measure on concatenated dictionaries give good accuracy.  The speaker and/or  the noise   belonging to unknown sources is handled by online adaptation.
			
			
			
		%	In the unsupervised acoustic-phonetics knowledge based approach, we detect			transitions between broad phonetic classes in a speech signal which has			applications such as landmark detection and segmentation. A rule-based			approach using relative thresholds learnt from a small development set is			devised to detect transitions of silence to non-silence, sonorant to			non-sonorant and vice-versa. This approach does not require significant			training data for determining the parameters of the proposed approach.We have also proposed subband analysis of linear prediction			residual (LPR) to estimate the GCIs from voiced speech segments.
			
			The \textbf{areas} and \textbf{techniques} explored in the research contributing to the thesis include the following:
			
			\vspace{-3.5mm}
			\begin{itemize}
				\item Audio Classification and Analysis pertaining to Speaker, Background Noise and Language Classification, Source Separation, Audio Signal Analysis, Audio Segmentation, Speech Enhancement, Event Detection 
				\item  \textbf{Machine learning techniques} used are feature extraction, feature selection, dictionary learning, model adaptation, semi-supervised learning, using validation set to learn the parameters, support vector machine (SVM), clustering, non-negative matrix factorization, linear and histogram based classifiers, heirarchical  and rule based classifiers
				
				\item \textbf{Signal processing techniques} like frame wise analysis, frequency domain analysis, linear prediction analysis, envelope extraction, extrema based analysis, bandpass filtering, sparse coding and non-negative sparse representations
			\end{itemize}
			
			\section{\mysidestyle Work Experience}
		
			\begin{list2}
			\item \emph{Data Scientist Tech Lead, Belong.co,}\hfill \textbf{June 2019- present}
			\begin{itemize}
			%	\item Published a paper on resume parsing and presented the same in IEEE ICDAR-WML in Sydney
			%	\item Documentation of the email classification and resume parsing algorithms
			\item Lead the team involved in building the Best Resume Parser in India right from data analysis, creating training data, modelling to deployment of the algorithms
				\item Technical breakdown, prioritization, future roadmap and guiding the resume parsing team 
				\item Innovating and devising new algorithms for improving resume parsing using statistical approaches, image processing, prior assumptions and deep learning based approaches  
				\item Planning, coding, deploying, bug fixing and fine tuning the resume parsing algorithm
			\end{itemize}
				
			\item \emph{Data Scientist, Belong.co,}\hfill \textbf{February 2018- May 2019}
			\begin{itemize}
				\item Data analyis, cleaning and pre-processing and co-ordinating with Data Operations team to get manually tagged data
				\item Devised sequence level primary and secondary labels for email classification  based on long term use cases
				\item Devised  the deep learning architecture for the email  classification model, coding, testing and deploying the same on production
				\item Devised and deployed unsupervised image processing based approaches for heading extraction and localize text boxes in resumes
				\item Devised semi-supervised algorithms and post-processing  to improve accuracy
			\end{itemize}
			\item \emph{Senior Technical Leader, Huawei Technologies India,} \hfill \textbf{April 2017- February 2018}
			
			\begin{itemize}
				\item Analysis and review of machine learning services on various cloud platforms \vspace{-.1cm}
			%	\item Brainstorming and  proposed patents on noise and audio analysis for cloud platform \vspace{-.1cm}
				\item  Artificial intelligence and machine learning resource analysis both in industry and academia
				\item Gave technical lectures and guidance on machine learning  concepts and algorithms and  mathematical background to  relevant teams
				\item  Created proposals for patents and getting new project on audio classification and noise analysis, speaker classification and voice assistant on Edge and Cloud platform %\vspace{-.1cm}
			%	\item Published conference paper on distributed audio classification in COMSNETS 2018 \vspace{-.1cm}
				\item Designed and developed a  system for face detection and recognition
				algorithms for IOT devices over Edge network single handedly using various feature extraction and supervised classification methods
				
			\end{itemize}
			
			\item \emph{Senior Engineer, Relay and Integrated
				Solutions, Larsen and Toubro Limited, Mumbai,} \hfill
			\textbf{August 2008- July 2010} 
			\begin{itemize}
				\item
				Development, testing, documentation and analysis of new product: Intelligent Motor Protection Relay, MCOMP 

				
			\end{itemize}
			
			
			
			
			\item
			\emph{Industrial Training:
				Inplant Practical Training at the Bharat Heavy
				Electricals Limited, Electronics Division, Bangalore,}
		\hfill	\textbf{May 2006-June 2006}
		\end{list2}
	
	
			%__________________________________________________________________________________________________________________
			% Education 
			\section{\mysidestyle Education}
			\vspace{1mm}
			\textbf{PhD, M.Sc(Engg)} \hfill \textbf{ August 2010 - September 2017 (Thesis Defense)}\vspace{-3mm}\\\vspace{-1mm}\\%
			Indian Institute of Science, Bangalore, India \vspace{0mm}\\%
			Advisors: Prof. A. G. Ramakrishnan, Electrical Engineering, IISc and Dr. T. V. Ananthapadmanabha, Voice and Speech Systems, Bangalore, India\vspace{-6mm}\\
			
			\textbf{Bachelor of Technology} \hfill \textbf{ August 2004-May 2008}\vspace{-3mm}\\\vspace{-1mm}\\%
			 Electrical and Electronics Engineering\\
			National Institute of Technology Karnataka, Surathkal, India \vspace{0mm}\\%
		%	CGPA: 7.75/10 \vspace{-6mm}\\
			
		%	\textbf{12th Board Schooling} \hfill \textbf{ 2001-2003}\vspace{-3mm}\\\vspace{-1mm}\\%
		%		Kendriya Vidyalaya No.-2, Kharagpur, India \vspace{0mm}\\%
		%	Percentage: 82\%, Central Board of Secondary Education\vspace{-6mm}\\
			
			%\textbf{10th Board Schooling} \hfill \textbf{ 2000-2001}\vspace{-3mm}\\\vspace{-1mm}\\%
			%	Kendriya Vidyalaya No.-2, Kharagpur, India \vspace{0mm}\\%Percentage: 76\%, Central Board of Secondary Education
			\vspace{-1.5mm}
			%__________________________________________________________________________________________________________________
			% Education 
			
		\section{\mysidestyle Relevant Academic Projects}	
		\vspace{.1cm}
 \textbf{ During PhD (apart from thesis)} \textit{August 2010- April 2017:}
\begin{itemize}\itemsep=0.25em
	\item
	Relationship Between Indian Languages Using Bigram Language Models- Group
project under Prof. T. V. Sreenivas of Dept of Electrical Communication Engineering,
Indian Institute of Science

\item
Voiced and Unvoiced feature classification of Speech data using multiple features-
Individual course project under Prof. A. G. Ramakrishnan of Dept of Electrical
Engineering, Indian Institute of Science

\item
Implementation of Modification of Pitch contour , Duration and Energy   for change in Prosody in Tamil Speech Synthesis under Prof. A. G. Ramakrishnan of Dept of Electrical
Engineering, Indian Institute of Science
\item
Voice Modification system using change of sampling frequency and duration normalization under Prof. A. G. Ramakrishnan of Dept of Electrical
Engineering, Indian Institute of Science
\item
Implementation of Parts of Speech and Pause Rules Tagging of Tamil text in Matlab and C under Prof. A. G. Ramakrishnan of Dept of Electrical
Engineering, Indian Institute of Science
\item Created interactive Matlab demos for voice modification, noise classfication and audio source separation for IISc Open Day


	\end{itemize}

 \textbf{ During B.Tech} \textit{August 2004 - May 2008:}
 
\begin{itemize}\itemsep=0.25em

%\item
%Design and implementation of Badminton Scoreboard (for doubles match) using Xilinx VHDL software- individual project under Prof. P.Vittal of Dept of Electrical and Electronics Engineering, NITK Surathkal

\item
Graphic Equalizer implementation using Matlab 7.0
- individual project under Prof. Jora M Gonda of Dept of
Electrical and Electronics Engineering, NITK Surathkal

%\item
%Design of a a simplified bus model of Embedded Intel486 SX Processor using Xilinx VHDL software- group project under Prof. P.Vittal of Dept of Electrical and Electronics Engineering, NITK Surathkal



\item
Implementation of Image Processing Techniques using Graphical User
Interface in Matlab 7.0.1- group project under Mrs. Vinatha U. of Dept of Electrical
and Electronics Engineering, NITK Surathkal





\end{itemize}		
			
			\section{\mysidestyle Additional Research Exposure}
			\vspace{1mm}
		
			
			\textbf{Project Associate} \hfill \textbf{ August 2016 -- April 2017}\vspace{-3mm}\\\vspace{-1mm}\\%
			 Project on \textit{Speaker and background change detection}, MILE Lab, IISc\\ Defence Research and Development Organization,
			Mentor: Prof.~A~G.~Ramakrishnan\vspace{-6mm}\\
			
		
			\vspace{-1.5mm}
			%__________________________________________________________________________________________________________________
			% TAs
			\section{\mysidestyle Teaching Assistantship}
			\vspace{1mm}
			%    \begin{list2}
		

		\textbf{Linear and Nonlinear Optimization} offered  by Prof. Muthuvel Arigovindan at  Indian Institute of Science, Bangalore during 
		August-December, 2013: Responsibilities included conducting tutorial classes  
		\vspace{-3mm}
		
		\textbf{Speech Information Processing}  offered  by Prof. A G Ramakrishnan at  Indian Institute of Science, Bangalore during 
		January-May, 2014: Responsibilities included   preparing assignments and projects
				\vspace{-3mm}
		
		\textbf{Matrix Theory}  offered  by Prof. A G Ramakrishnan at  Indian Institute of Science, Bangalore during 
		August-December, 2014: Responsibilities included   preparing assignments, clearing doubts, evaluation and grading of students.
		\vspace{1mm}
			%    \end{list2}
			\vspace{-1.5mm}
			%__________________________________________________________________________________________________________________
			% PMs
			\section{\mysidestyle Research Mentorship}
			\vspace{1mm}
			%    \begin{list2}
		\noindent	\textbf{ May-July, 2016}  Veena Vijai, from Birla Institute of Technology and Science, Pilani - K. K. Birla Goa Campus  on Relationship between spoken Indian languages by clustering of long distance bigram features of speech	
				\vspace{-3mm}
				
			\textbf{ June-August, 2016} B Shubashree, from SSN College of Engineering, Chennai on Unsupervised background noise change identification using dictionary learning
			%    \end{list2}
			\vspace{-1.5mm}
			
			% Courses Crdeited
			\vspace{-2.5mm}
			\section{\mysidestyle Graduate Courses} 
			\vspace{2.5mm}
		\textbf{Credited}:	Matrix Theory,
			Linear and Non Linear Optimization, Stochhastic Models and Applications,		
			Pattern Recognition and Neural Networks, Data Mining, Compressive Sensing and Sparse Signal Processing, 	Advanced Digital Signal Processing, Speech Information Processing, Automatic Speech Recognition Algorithms\\
			\textbf{Audited: } Time Frequency Analysis, Convex Optimization , Digital Image Processing, Machine Learning
			
						\section{\mysidestyle Online Courses} 
						Pattern Recognition (\href{http://nptel.ac.in/courses/117108048/#}{NPTEL}), Machine Learning (\href{https://www.coursera.org/learn/machine-learning/home/welcome}{Coursera}), Natural Language Processing ( \href{https://www.youtube.com/playlist?list=PL6397E4B26D00A269}{Dan Jurafsky}), Introduction to Python for Data Science (\href{https://courses.edx.org/courses/course-v1:Microsoft+DAT208x+2T2017/course/}{edx}),  Neural Networks and Deep Learning by deeplearning.ai on Coursera: Certificate earned on Monday, February 5, 2018 3:11 AM GMT (\href{https://www.coursera.org/account/accomplishments/records/QSWQJRUQ47AX}{Coursera})
						
						
			%__________________________________________________________________________________________________________________ 
			% Computer Skills
			\vspace{-3.5mm}
			\section{\mysidestyle Programming Languages and Computer Skills} 
			\vspace{2.5mm}
			 Python, Matlab, C \vspace{1mm}\\
			Operating system usage UNIX and MacOS \vspace{1mm}\\
			Report documentation with \LaTeXe
			\vspace{-2.5mm}
			% Publications
			\section{\mysidestyle Publications}
			\vspace{1.5mm}
			Conference  and Workshop papers:
			\begin{list2}
			\item Vinodh Kumar Ravindranath, Devashish Deshpande, K V Vijay Girish, Darshan Patel and Neel Jambhekar
Vikash Singh, \textit{ Inferring Structure and Meaning of Semi-Structured Documents by using a Gibbs Sampling Based Approach}, IEEE ICDAR-WML 2019, Sydney
					\item K V Vijay Girish, A G Ramakrishnan and Neeraj Kumar, \textit{A system for distributed audio classification using sparse representation over cloud for IOT}, IEEE COMSNETS 2018, Bangalore
			
			\item K V Vijay Girish and A G Ramakrishnan, \textit{Enhancement of noisy Tamil speech for improved quality of perception for the hearing impaired}, 16th Tamil Internet Conference 2017, Toronto
					
				\item  K V Vijay Girish, T V Ananthapadmanabha and A G Ramakrishnan,  \textit{Cosine similarity based dictionary learning and source recovery for classification of diverse audio sources}, IEEE INDICON 2016, IISc Bangalore
					
				
				\item  K V Vijay Girish, Veena Vijai and A G Ramakrishnan,  \textit{Relationship between spoken Indian languages by clustering of long distance bigram features of speech},  IEEE INDICON 2016, IISc Bangalore
				
				\item	K V Vijay Girish, A G Ramakrishnan and T V Ananthapadmanabha, \textit{Hierarchical classification of speaker and background noise and estimation of SNR using sparse representation}, Interspeech 2016, September 8-12, 2016, San Francisco
				
				
				
				\item Vikram R L, K V Vijay Girish, Harshavardhan S, A G Ramakrishnan, T V Ananthapadmanabha, 
				\textit{Subband Analysis of Linear Prediction Residual for the Estimation of Glottal Closure Instants}, Proc. IEEE International Conference on Acoustics, Speech and Signal Processing (ICASSP 2014), May 4-9, 2014, Florence, Italy
				
				
				\item
				Vikram Ramesh Lakkavalli, K V Vijay Girish, A G Ramakrishnan, \textit{Sub-band Envelope Approach to Obtain Instants of Significant Excitation in Speech}, Proc. National Conference on Communications (NCC 2012), Feb 3-5, 2012, Kharagpur, India, pp. 19
				
				
				\item Sayan Ghosh, K V Vijay Girish, T V Sreenivas, \textit{Relationship between Indian Languages Using Long Distance Bigram Language Models}, Proc. International Conference on Natural Language Processing (ICON
			2011), Dec 16-19, 2011, Chennai, India, pp. 104-113
			
			
			\end{list2}
			\vspace{-1mm}
			Journal paper:
			\begin{list2}
			\item T V Ananthapadmanabha, K V Vijay Girish and A G Ramakrishnan,  \textit{Relative occurrences and difference of extrema for detection of transitions between broad phonetic classes},  Sadhana
			 (2018) 43: 153. \url{https://doi.org/10.1007/s12046-018-0923-x}
			\end{list2}
		
			\vspace{-1mm}
		Technical Reports:
		\begin{list2}
		\item T V Ananthapadmanabha, K V Vijay Girish, A G Ramakrishnan, \textit{Detection of transitions between broad phonetic classes in a speech signal}, 	arXiv:1411.0370 [cs.SD]
		%\item K V Vijay Girish, T V Ananthapadmanabha, A G Ramakrishnan, \textit{A dictionary learning and source recovery based approach to classify diverse audio sources}, arXiv, Submitted on 27 Oct 2015
		
		\item  K V Vijay Girish, A G Ramakrishnan and T V Ananthapadmanabha,  \textit{Adaptive dictionary based approach for background noise and speaker classification and subsequent source separation}, 	arXiv:1609.09764 [cs.SD]
		\end{list2}
			%__________________________________________________________________________________________________________________
			% Technical Schools
			\vspace{-2.5mm}
			\section{\mysidestyle Talks and Poster Presentations} 
			\vspace{5mm}
			\begin{list2}
			\item Gave an IEEE talk on ``Throwing light on sound" in NIT Goa and BITS Goa, on 12th February, 2017
			\item Presented poster on ``Adaptive and supervised sparse representation based approach for noisy speech analysis"  in IISconnect: Industry Interaction Day at IISc  Bangalore on 3rd October, 2016 
			
			\item Gave a talk and presented poster on ``Analysis of audio intercepts: Can we identify and locate the speaker? " at EECS Research Students Symposium - 2016,  held at IISc Bangalore during 28-29 April, 2016
			
			
			\item 
			Presented poster on my accepted paper in INTERSPEECH-2016 at San Francisco, USA during 8-12 September, 2016
			
			
			\item Presented poster on my accepted paper in IEEE International Conference on Acoustics, Speech and Signal Processing (ICASSP 2014) at Florence, Italy  during  4-9 May, 2014
			
			\item 
			
			Gave a talk on my accepted paper in  National Conference on Communications,  held at IIT Kharagpur, during 3-5 February, 2012
			
			\item Assisted Prof. A. G. Ramakrishnan in conducting a tutorial on Insights into Signal Processing,Transforms and Linear Algebra at International Conference on Biomedical Engineering, 2011 held at MIT Manipal, during 8-9 December, 2011
			
			
			
			
			
			
			\end{list2}
			%__________________________________________________________________________________________________
			%__________________________________________________________________________________________________________________
			% Hobbies
			
	%__________________________________________________________________________________________________________________    
	
	\vspace{-1.5mm}
	%__________________________________________________________________________________________________________________    
	% Honours and Awards
	%	\section{\mysidestyle Grants}
			%__________________________________________________________________________________________________________________
			% Hobbies
			\vspace{-3.5mm}
			\section{\mysidestyle Technical Activities}
			\vspace{4.5mm}
			\begin{list2}
				\item \textbf{Reviewer} of reputed conferences/ journals like Annual ISCA Interspeech conference,  IEEE Signal Processing and Communication (SPCOM) Conference, and IEEE Transactions on Signal Processing
				\item Organization Committee member in Annual Electrical Sciences Divisional Symposium, 2013, IISc, an annual symposium for graduating students of IISc to talk
				about their findings and collaborate with industries.
				\item \textbf{Writing} technical blogs in Medium: https://medium.com/@girish.vijay
				\item Showcased Interactive Matlab Demos in Open Day 2012,  2014, 2015, 2016 and 2017, an annual one day event hosted at Indian Institute of Science, Bangalore to make public familiar with activities at the institute.	\item Participated in Google Code Jam regularly
				
				\item
				Bagged 2nd prize in Foxhunt in the TechFest Engineer 2006
				conducted by NITK Surathkal
				\item
				Secured 5th position in Simplicity, an International
				Online Matlab Programming contest conducted during
				Engineer 2008, a Technical Festival conducted by NITK
				Surathkal
				
				\item Bagged 3rd prize in Science Slam in Pravega Sci-Tech 2014 conducted by Swissnex at IISc Bangalore
				
			\end{list2}
		
	% Honours and Awards
	\section{\mysidestyle Awards and Achievements}
	\vspace{1mm}
Awarded 40 Under 40 Data Scientists at MLDS 2020, organized by Analytics India Magazine \hfill \textbf{2020} \vspace{1mm} \\
	Secured an All India Rank of 4525 and State Rank (West Bengal) of 53 in AIEEE 2004 \hfill \textbf{2004} \vspace{1mm}\\% 
	Secured  98.8 percentile in GATE 2010 (Electrical Engineering) \hfill \textbf{2010} \vspace{1mm}\\%
	%    IEEE Signal Processing Society Student Travel Grant to attend ICASSP held in Kyoto, 2012. \vspace{1mm}\\%
	PhD Scholarship, Ministry of Human Resource and Development, Govt. of India \hfill \textbf{2010-2016} \vspace{1mm}\\%
	Distinctive performance in National Science Olympiad\hfill \textbf{2000}                                      
		
	\section{\mysidestyle Extra-curricular activities}
			
			\begin{list2}
			
				\item
				
				
				Active Member of  Spicmacay, Voice club, IEEE and Management Forum
				at NITK Surathkal
				\item Active member of Mess committee at IISc Bangalore from July 2013 - May 2015
				\item
				
				Playing badminton,  running, swimming, biking, triathlon,  programming, technology, guitar,  photography, writing blogs
				
				\item Finished Ironman 70.3 in Hyderabad triathlon and Olympic triathlon in Goa triathlon events in 2017
				
				
		\end{list2}
	
				\section{\mysidestyle Languages known}
				English, Hindi and Telugu 
			%__________________________________________________________________________________________________________________
			%     \section{\mysidestyle Referees} 
			%     {\sl Available on request.}
			% %______________________________________________________________________________________________________________________
			
			% \vspace{-4.5mm}
			\section{\mysidestyle References} 
        \textsl{Available on request}  
%	\begin{tabular}{@{}p{6cm}p{6cm}}
				% \textbf{Professor Thippur V. Sreenivas}       &  \textbf{Dr Stephen So}                   \\
				% Professor, Dept. ECE                          &  Associate Lecturer                       \\
				% Indian Institute of Science             &  Griffith University                      \\
				% Banglore, India                         &  Gold Coast, Queensland, Australia        \\
				% % phone: \textsl{available on request}    &  phone: \textsl{available on request}     \\
				% % e-mail: \textsl{available on request}   &  e-mail: \textsl{available on request}    \\
				
			%	\vspace{6mm}
			%	\textbf{Vinodh Kumar Ravindranath}                         \\
			%	\vspace{.01mm}
			%	Chief Technology Officer                                             \\
			%	Belong.co                                     \\
			%%	Banglore, India                                                 \\
				% phone: \textsl{available on request}    &  phone: \textsl{available on request}     \\
			%	e-mail: \textsl{vinodhr@gmail.com}  \\
			%	Linkedin: \textsl{https://www.linkedin.com/in/vinodhkumarr/}\\
				
			%	\vspace{6mm}
			%	\textbf{Prof. A G Ramakrishnan}                         \\
			%	\vspace{.01mm}
			%%	Professor, Dept. EE                                            \\
		%		Indian Institute of Science                                     \\
		%		Banglore, India                                                 \\
				% phone: \textsl{available on request}    &  phone: \textsl{available on request}     \\
		%		e-mail: \textsl{agr@iisc.ac.in}  \\
				
				
		%		\vspace{6mm}
		%		\textbf{Dr. T V Ananthapadmanabha}                         \\
		%		\vspace{.01mm}
	%			Founder and Chief innovator,\\
		%		Voice and Speech Systems, Bangalore, India\\                                          
				% phone: \textsl{available on request}    &  phone: \textsl{available on request}     \\
		%		e-mail: \textsl{tva.blr@gmail.com}  \\
				
				
				
				
				%\vspace{6mm}
				%\textbf{Prof. Thippur V. Sreenivas}                         \\
				%Professor, Dept. ECE                                            \\
				%Indian Institute of Science                                     \\
				%Banglore, India                                                 \\
				% phone: \textsl{available on request}    &  phone: \textsl{available on request}     \\
				%e-mail: \textsl{tvsree@iisc.ac.in}  \\
				
				
				
				% \vspace{6mm}
				% \textbf{K.V.S. Hari}                         \\
				% Professor, Dept. ECE                                   \\
				% Indian Institute of Science                                     \\
				% Banglore, India                                                 \\
				% phone: \textsl{available on request}    &  phone: \textsl{available on request}     \\
				% e-mail: \textsl{available on request}   &  e-mail: \textsl{available on request}    \\
		%	\end{tabular}
			
			%______________________________________________________________________________________________________________________
		\end{resume}
		
	\end{document}
	
	
	%______________________________________________________________________________________________________________________
	% EOF
	
%______________________________________________________________________________________________________________________
% @brief    LaTeX2e Resume for Kamil K Wojcicki
\documentclass[line]{resume}
\usepackage{hyperref}
%\usepackage[paperwidth=25cm,paperheight=22cm,left=1cm,top=1cm]{geometry}
\usepackage[paperwidth=21cm,paperheight=30cm,left=2cm,top=1.5cm]{geometry}
\renewcommand{\familydefault}{\sfdefault}
\fontfamily{garamond}
%______________________________________________________________________________________________________________________
\begin{document}
	\small{
		\name{ \Large K V Vijay Girish}
		\begin{resume}
			\setcounter{page}{1}
			\pagenumbering{arabic} 
			%     \vspace{-1.5mm}
			%      PhD Student
			\vspace{-4.5mm}
			%__________________________________________________________________________________________________________________
			% Contact Information
			%    \section{\mysidestyle Contact\\Information}
			\vspace{1mm}
		 Data Scientist Tech Lead          					\hfill e-mail: vijay.girish@gmail.com \\
			Belong.co,   \hfill Website: \url{https://sites.google.com/site/kvvijaygirish/}\\
			%    Electrical Engineering 	                \\
		   Bangalore, INDIA  \hfill Linkedin: \url{https://www.linkedin.com/in/k-v-vijay-girish-b85a3714/}         \\			
		 \hspace{5cm}  Mobile no.: +91 9480515318 \hfill Github: \url{https://github.com/vijaygirish2001}
		 %\vspace{-1.5mm}%
			%__________________________________________________________________________________________________________________
			%     \section{\mysidestyle Date of Birth}
			%     11 June 1987,  Age: 30
			%     \vspace{-2.5mm}
			%  \section{\mysidestyle Goal} To work for an internship in a dynamic environment exploring an exciting area in signal processing
			% applications and DSP coding. 
			%__________________________________________________________________________________________________________________
			%     % Current Research Work
			%\vspace{-2.5mm}
			\section{\mysidestyle Research Interests}
			\vspace{1mm}
			I am  interested in designing innovative mathematical and machine learning models to analyze, process and extract  information from signals and data  captured from various domains like audio, vision and text. I enjoy analytical and emperical modelling, and  out of the box thinking to solve practical problems. 
			My broad areas of interest are:\vspace{-.1cm}
			
			 Artificial Intelligence, Machine Learning, Deep learning, Data Analytics, Signal Processing,  Sparse Representations, Dictionary Learning and Adaptation, Speech and Audio Analysis, Source Separation, Acoustic Noise and Speaker Classification,  Speech Recognition, Keyword Spotting, Image and Video  Processing
			\vspace{-.1cm}
			
			I am especially passionate about audio and speech research, and their applications to real world problems.
      	\section{\mysidestyle PhD Thesis Summary}
      	
      	\textbf{Thesis title:} Speech and noise analysis using sparse representation and acoustic phonetics knowledge\\	
			\noindent\textbf{Abstract}  This thesis  addresses  different  aspects of speech and noise analysis using two different approaches:\vspace{-.5cm}
			\begin{enumerate}
				\item  A supervised and adaptive sparse representation based approach for identifying the type of background noise and the speaker and separating the speech and background noise, and
				\item An unsupervised acoustic-phonetics knowledge  based approach for detecting transitions between broad phonetic classes  and significant excitation instants called as glottal closure instants (GCIs) in a speech signal, for applications like speech segmentation, recognition and modification
			\end{enumerate} 
			
			
		%	In the supervised sparse representation based approach, a dictionary			learning based noise classification algorithm is proposed using a cosine			similarity measure for learning atoms of the dictionary. We have used the			Active Set Newton Algorithm (ASNA) and supervised non-negative matrix factorization  for source recovery in the testing			phase. 	For speaker classification on clean speech,  using  high energy subsets of test frames and dictionary atoms, sum of weights measure on concatenated dictionaries give good accuracy.  The speaker and/or  the noise   belonging to unknown sources is handled by online adaptation.
			
			
			
		%	In the unsupervised acoustic-phonetics knowledge based approach, we detect			transitions between broad phonetic classes in a speech signal which has			applications such as landmark detection and segmentation. A rule-based			approach using relative thresholds learnt from a small development set is			devised to detect transitions of silence to non-silence, sonorant to			non-sonorant and vice-versa. This approach does not require significant			training data for determining the parameters of the proposed approach.We have also proposed subband analysis of linear prediction			residual (LPR) to estimate the GCIs from voiced speech segments.
			
			The \textbf{areas} and \textbf{techniques} explored in the research contributing to the thesis include the following:
			
			\vspace{-3.5mm}
			\begin{itemize}
				\item Audio Classification and Analysis pertaining to Speaker, Background Noise and Language Classification, Source Separation, Audio Signal Analysis, Audio Segmentation, Speech Enhancement, Event Detection 
				\item  \textbf{Machine learning techniques} used are feature extraction, feature selection, dictionary learning, model adaptation, semi-supervised learning, using validation set to learn the parameters, support vector machine (SVM), clustering, non-negative matrix factorization, linear and histogram based classifiers, heirarchical  and rule based classifiers
				
				\item \textbf{Signal processing techniques} like frame wise analysis, frequency domain analysis, linear prediction analysis, envelope extraction, extrema based analysis, bandpass filtering, sparse coding and non-negative sparse representations
			\end{itemize}
			
			\section{\mysidestyle Work Experience}
		
			\begin{list2}
			\item \emph{Data Scientist Tech Lead, Belong.co,}\hfill \textbf{June 2019- present}
			\begin{itemize}
			%	\item Published a paper on resume parsing and presented the same in IEEE ICDAR-WML in Sydney
			%	\item Documentation of the email classification and resume parsing algorithms
				\item Technical breakdown, prioritization, future roadmap and guiding the resume parsing team
				\item Devising new algorithms (like table extraction and section detection) for improving resume parsing using statistical approaches, image processing, prior assumptions and deep learning based approaches  
				\item Devising, coding, deploying, bug fixing and fine tuning the resume parsing algorithm
			\end{itemize}
				
			\item \emph{Data Scientist, Belong.co,}\hfill \textbf{February 2018- May 2019}
			\begin{itemize}
				\item Data analyis, cleaning and pre-processing and co-ordinating with Data Operations team to get manually tagged data
				\item Devised sequence level primary and secondary labels for email classification  based on long term use cases
				\item Devised  the deep learning architecture for the email  classification model, coding, testing and deploying the same on production
				\item Devised and deployed unsupervised image processing based approaches for heading extraction and localize text boxes in resumes
				\item Devised semi-supervised algorithms and post-processing  to improve accuracy
			\end{itemize}
			\item \emph{Senior Technical Leader, Huawei Technologies India,} \hfill \textbf{April 2017- February 2018}
			
			\begin{itemize}
				\item Analysis and review of machine learning services on various cloud platforms \vspace{-.1cm}
			%	\item Brainstorming and  proposed patents on noise and audio analysis for cloud platform \vspace{-.1cm}
				\item  Artificial intelligence and machine learning resource analysis both in industry and academia
				\item Gave technical lectures and guidance on machine learning  concepts and algorithms and  mathematical background to  relevant teams
				\item  Created proposals for patents and getting new project on audio classification and noise analysis, speaker classification and voice assistant on Edge and Cloud platform %\vspace{-.1cm}
			%	\item Published conference paper on distributed audio classification in COMSNETS 2018 \vspace{-.1cm}
				\item Designed and developed a  system for face detection and recognition
				algorithms for IOT devices over Edge network single handedly using various feature extraction and supervised classification methods
				
			\end{itemize}
			
			\item \emph{Senior Engineer, Relay and Integrated
				Solutions, Larsen and Toubro Limited, Mumbai,} \hfill
			\textbf{August 2008- July 2010} 
			\begin{itemize}
				\item
				Development, testing, documentation and analysis of new product: Intelligent Motor Protection Relay, MCOMP 

				
			\end{itemize}
			
			
			
			
			\item
			\emph{Industrial Training:
				Inplant Practical Training at the Bharat Heavy
				Electricals Limited, Electronics Division, Bangalore,}
		\hfill	\textbf{May 2006-June 2006}
		\end{list2}
	
	
			%__________________________________________________________________________________________________________________
			% Education 
			\section{\mysidestyle Education}
			\vspace{1mm}
			\textbf{PhD, M.Sc(Engg)} \hfill \textbf{ August 2010 - September 2017 (Thesis Defense)}\vspace{-3mm}\\\vspace{-1mm}\\%
			Indian Institute of Science, Bangalore, India \vspace{0mm}\\%
			Advisors: Prof. A. G. Ramakrishnan, Electrical Engineering, IISc and Dr. T. V. Ananthapadmanabha, Voice and Speech Systems, Bangalore, India\vspace{-6mm}\\
			
			\textbf{Bachelor of Technology} \hfill \textbf{ August 2004-May 2008}\vspace{-3mm}\\\vspace{-1mm}\\%
			 Electrical and Electronics Engineering\\
			National Institute of Technology Karnataka, Surathkal, India \vspace{0mm}\\%
		%	CGPA: 7.75/10 \vspace{-6mm}\\
			
		%	\textbf{12th Board Schooling} \hfill \textbf{ 2001-2003}\vspace{-3mm}\\\vspace{-1mm}\\%
		%		Kendriya Vidyalaya No.-2, Kharagpur, India \vspace{0mm}\\%
		%	Percentage: 82\%, Central Board of Secondary Education\vspace{-6mm}\\
			
			%\textbf{10th Board Schooling} \hfill \textbf{ 2000-2001}\vspace{-3mm}\\\vspace{-1mm}\\%
			%	Kendriya Vidyalaya No.-2, Kharagpur, India \vspace{0mm}\\%Percentage: 76\%, Central Board of Secondary Education
			\vspace{-1.5mm}
			%__________________________________________________________________________________________________________________
			% Education 
			
		\section{\mysidestyle Relevant Academic Projects}	
		\vspace{.1cm}
 \textbf{ During PhD (apart from thesis)} \textit{August 2010- April 2017:}
\begin{itemize}\itemsep=0.25em
	\item
	Relationship Between Indian Languages Using Bigram Language Models- Group
project under Prof. T. V. Sreenivas of Dept of Electrical Communication Engineering,
Indian Institute of Science

\item
Voiced and Unvoiced feature classification of Speech data using multiple features-
Individual course project under Prof. A. G. Ramakrishnan of Dept of Electrical
Engineering, Indian Institute of Science

\item
Implementation of Modification of Pitch contour , Duration and Energy   for change in Prosody in Tamil Speech Synthesis under Prof. A. G. Ramakrishnan of Dept of Electrical
Engineering, Indian Institute of Science
\item
Voice Modification system using change of sampling frequency and duration normalization under Prof. A. G. Ramakrishnan of Dept of Electrical
Engineering, Indian Institute of Science
\item
Implementation of Parts of Speech and Pause Rules Tagging of Tamil text in Matlab and C under Prof. A. G. Ramakrishnan of Dept of Electrical
Engineering, Indian Institute of Science
\item Created interactive Matlab demos for voice modification, noise classfication and audio source separation for IISc Open Day


	\end{itemize}

 \textbf{ During B.Tech} \textit{August 2004 - May 2008:}
 
\begin{itemize}\itemsep=0.25em

%\item
%Design and implementation of Badminton Scoreboard (for doubles match) using Xilinx VHDL software- individual project under Prof. P.Vittal of Dept of Electrical and Electronics Engineering, NITK Surathkal

\item
Graphic Equalizer implementation using Matlab 7.0
- individual project under Prof. Jora M Gonda of Dept of
Electrical and Electronics Engineering, NITK Surathkal

%\item
%Design of a a simplified bus model of Embedded Intel486 SX Processor using Xilinx VHDL software- group project under Prof. P.Vittal of Dept of Electrical and Electronics Engineering, NITK Surathkal



\item
Implementation of Image Processing Techniques using Graphical User
Interface in Matlab 7.0.1- group project under Mrs. Vinatha U. of Dept of Electrical
and Electronics Engineering, NITK Surathkal





\end{itemize}		
			
			\section{\mysidestyle Additional Research Exposure}
			\vspace{1mm}
		
			
			\textbf{Project Associate} \hfill \textbf{ August 2016 -- April 2017}\vspace{-3mm}\\\vspace{-1mm}\\%
			 Project on \textit{Speaker and background change detection}, MILE Lab, IISc\\ Defence Research and Development Organization,
			Mentor: Prof.~A~G.~Ramakrishnan\vspace{-6mm}\\
			
		
			\vspace{-1.5mm}
			%__________________________________________________________________________________________________________________
			% TAs
			\section{\mysidestyle Teaching Assistantship}
			\vspace{1mm}
			%    \begin{list2}
		

		\textbf{Linear and Nonlinear Optimization} offered  by Prof. Muthuvel Arigovindan at  Indian Institute of Science, Bangalore during 
		August-December, 2013: Responsibilities included conducting tutorial classes  
		\vspace{-3mm}
		
		\textbf{Speech Information Processing}  offered  by Prof. A G Ramakrishnan at  Indian Institute of Science, Bangalore during 
		January-May, 2014: Responsibilities included   preparing assignments and projects
				\vspace{-3mm}
		
		\textbf{Matrix Theory}  offered  by Prof. A G Ramakrishnan at  Indian Institute of Science, Bangalore during 
		August-December, 2014: Responsibilities included   preparing assignments, clearing doubts, evaluation and grading of students.
		\vspace{1mm}
			%    \end{list2}
			\vspace{-1.5mm}
			%__________________________________________________________________________________________________________________
			% PMs
			\section{\mysidestyle Research Mentorship}
			\vspace{1mm}
			%    \begin{list2}
		\noindent	\textbf{ May-July, 2016}  Veena Vijai, from Birla Institute of Technology and Science, Pilani - K. K. Birla Goa Campus  on Relationship between spoken Indian languages by clustering of long distance bigram features of speech	
				\vspace{-3mm}
				
			\textbf{ June-August, 2016} B Shubashree, from SSN College of Engineering, Chennai on Unsupervised background noise change identification using dictionary learning
			%    \end{list2}
			\vspace{-1.5mm}
			
			% Courses Crdeited
			\vspace{-2.5mm}
			\section{\mysidestyle Graduate Courses} 
			\vspace{2.5mm}
		\textbf{Credited}:	Matrix Theory,
			Linear and Non Linear Optimization, Stochhastic Models and Applications,		
			Pattern Recognition and Neural Networks, Data Mining, Compressive Sensing and Sparse Signal Processing, 	Advanced Digital Signal Processing, Speech Information Processing, Automatic Speech Recognition Algorithms\\
			\textbf{Audited: } Time Frequency Analysis, Convex Optimization , Digital Image Processing, Machine Learning
			
						\section{\mysidestyle Online Courses} 
						Pattern Recognition (\href{http://nptel.ac.in/courses/117108048/#}{NPTEL}), Machine Learning (\href{https://www.coursera.org/learn/machine-learning/home/welcome}{Coursera}), Natural Language Processing ( \href{https://www.youtube.com/playlist?list=PL6397E4B26D00A269}{Dan Jurafsky}), Introduction to Python for Data Science (\href{https://courses.edx.org/courses/course-v1:Microsoft+DAT208x+2T2017/course/}{edx}),  Neural Networks and Deep Learning by deeplearning.ai on Coursera: Certificate earned on Monday, February 5, 2018 3:11 AM GMT (\href{https://www.coursera.org/account/accomplishments/records/QSWQJRUQ47AX}{Coursera})
						
						
			%__________________________________________________________________________________________________________________ 
			% Computer Skills
			\vspace{-3.5mm}
			\section{\mysidestyle Programming Languages and Computer Skills} 
			\vspace{2.5mm}
			 Python, Matlab, C \vspace{1mm}\\
			Operating system usage UNIX and MacOS \vspace{1mm}\\
			Report documentation with \LaTeXe
			\vspace{-2.5mm}
			% Publications
			\section{\mysidestyle Publications}
			\vspace{1.5mm}
			Conference  and Workshop papers:
			\begin{list2}
			\item Vinodh Kumar Ravindranath, Devashish Deshpande, K V Vijay Girish, Darshan Patel and Neel Jambhekar
Vikash Singh, \textit{ Inferring Structure and Meaning of Semi-Structured Documents by using a Gibbs Sampling Based Approach}, IEEE ICDAR-WML 2019, Sydney
					\item K V Vijay Girish, A G Ramakrishnan and Neeraj Kumar, \textit{A system for distributed audio classification using sparse representation over cloud for IOT}, IEEE COMSNETS 2018, Bangalore
			
			\item K V Vijay Girish and A G Ramakrishnan, \textit{Enhancement of noisy Tamil speech for improved quality of perception for the hearing impaired}, 16th Tamil Internet Conference 2017, Toronto
					
				\item  K V Vijay Girish, T V Ananthapadmanabha and A G Ramakrishnan,  \textit{Cosine similarity based dictionary learning and source recovery for classification of diverse audio sources}, IEEE INDICON 2016, IISc Bangalore
					
				
				\item  K V Vijay Girish, Veena Vijai and A G Ramakrishnan,  \textit{Relationship between spoken Indian languages by clustering of long distance bigram features of speech},  IEEE INDICON 2016, IISc Bangalore
				
				\item	K V Vijay Girish, A G Ramakrishnan and T V Ananthapadmanabha, \textit{Hierarchical classification of speaker and background noise and estimation of SNR using sparse representation}, Interspeech 2016, September 8-12, 2016, San Francisco
				
				
				
				\item Vikram R L, K V Vijay Girish, Harshavardhan S, A G Ramakrishnan, T V Ananthapadmanabha, 
				\textit{Subband Analysis of Linear Prediction Residual for the Estimation of Glottal Closure Instants}, Proc. IEEE International Conference on Acoustics, Speech and Signal Processing (ICASSP 2014), May 4-9, 2014, Florence, Italy
				
				
				\item
				Vikram Ramesh Lakkavalli, K V Vijay Girish, A G Ramakrishnan, \textit{Sub-band Envelope Approach to Obtain Instants of Significant Excitation in Speech}, Proc. National Conference on Communications (NCC 2012), Feb 3-5, 2012, Kharagpur, India, pp. 19
				
				
				\item Sayan Ghosh, K V Vijay Girish, T V Sreenivas, \textit{Relationship between Indian Languages Using Long Distance Bigram Language Models}, Proc. International Conference on Natural Language Processing (ICON
			2011), Dec 16-19, 2011, Chennai, India, pp. 104-113
			
			
			\end{list2}
			\vspace{-1mm}
			Journal paper:
			\begin{list2}
			\item T V Ananthapadmanabha, K V Vijay Girish and A G Ramakrishnan,  \textit{Relative occurrences and difference of extrema for detection of transitions between broad phonetic classes},  Sadhana
			 (2018) 43: 153. \url{https://doi.org/10.1007/s12046-018-0923-x}
			\end{list2}
		
			\vspace{-1mm}
		Technical Reports:
		\begin{list2}
		\item T V Ananthapadmanabha, K V Vijay Girish, A G Ramakrishnan, \textit{Detection of transitions between broad phonetic classes in a speech signal}, 	arXiv:1411.0370 [cs.SD]
		%\item K V Vijay Girish, T V Ananthapadmanabha, A G Ramakrishnan, \textit{A dictionary learning and source recovery based approach to classify diverse audio sources}, arXiv, Submitted on 27 Oct 2015
		
		\item  K V Vijay Girish, A G Ramakrishnan and T V Ananthapadmanabha,  \textit{Adaptive dictionary based approach for background noise and speaker classification and subsequent source separation}, 	arXiv:1609.09764 [cs.SD]
		\end{list2}
			%__________________________________________________________________________________________________________________
			% Technical Schools
			\vspace{-2.5mm}
			\section{\mysidestyle Talks and Poster Presentations} 
			\vspace{5mm}
			\begin{list2}
			\item Gave an IEEE talk on ``Throwing light on sound" in NIT Goa and BITS Goa, on 12th February, 2017
			\item Presented poster on ``Adaptive and supervised sparse representation based approach for noisy speech analysis"  in IISconnect: Industry Interaction Day at IISc  Bangalore on 3rd October, 2016 
			
			\item Gave a talk and presented poster on ``Analysis of audio intercepts: Can we identify and locate the speaker? " at EECS Research Students Symposium - 2016,  held at IISc Bangalore during 28-29 April, 2016
			
			
			\item 
			Presented poster on my accepted paper in INTERSPEECH-2016 at San Francisco, USA during 8-12 September, 2016
			
			
			\item Presented poster on my accepted paper in IEEE International Conference on Acoustics, Speech and Signal Processing (ICASSP 2014) at Florence, Italy  during  4-9 May, 2014
			
			\item 
			
			Gave a talk on my accepted paper in  National Conference on Communications,  held at IIT Kharagpur, during 3-5 February, 2012
			
			\item Assisted Prof. A. G. Ramakrishnan in conducting a tutorial on Insights into Signal Processing,Transforms and Linear Algebra at International Conference on Biomedical Engineering, 2011 held at MIT Manipal, during 8-9 December, 2011
			
			
			
			
			
			
			\end{list2}
			%__________________________________________________________________________________________________
			%__________________________________________________________________________________________________________________
			% Hobbies
			
	%__________________________________________________________________________________________________________________    
	
	\vspace{-1.5mm}
	%__________________________________________________________________________________________________________________    
	% Honours and Awards
	%	\section{\mysidestyle Grants}
			%__________________________________________________________________________________________________________________
			% Hobbies
			\vspace{-3.5mm}
			\section{\mysidestyle Technical Activities}
			\vspace{4.5mm}
			\begin{list2}
				\item \textbf{Reviewer} of reputed conferences/ journals like Annual ISCA Interspeech conference,  IEEE Signal Processing and Communication (SPCOM) Conference, and IEEE Transactions on Signal Processing
				\item Organization Committee member in Annual Electrical Sciences Divisional Symposium, 2013, IISc, an annual symposium for graduating students of IISc to talk
				about their findings and collaborate with industries.
				\item \textbf{Writing} technical blogs in Medium: https://medium.com/@girish.vijay
				\item Showcased Interactive Matlab Demos in Open Day 2012,  2014, 2015, 2016 and 2017, an annual one day event hosted at Indian Institute of Science, Bangalore to make public familiar with activities at the institute.	\item Participated in Google Code Jam regularly
				
				\item
				Bagged 2nd prize in Foxhunt in the TechFest Engineer 2006
				conducted by NITK Surathkal
				\item
				Secured 5th position in Simplicity, an International
				Online Matlab Programming contest conducted during
				Engineer 2008, a Technical Festival conducted by NITK
				Surathkal
				
				\item Bagged 3rd prize in Science Slam in Pravega Sci-Tech 2014 conducted by Swissnex at IISc Bangalore
				
			\end{list2}
		
	% Honours and Awards
	\section{\mysidestyle Awards and Achievements}
	\vspace{1mm}
Awarded 40 Under 40 Data Scientists at MLDS 2020, organized by Analytics India Magazine \hfill \textbf{2020} \vspace{1mm} \\
	Secured an All India Rank of 4525 and State Rank (West Bengal) of 53 in AIEEE 2004 \hfill \textbf{2004} \vspace{1mm}\\% 
	Secured  98.8 percentile in GATE 2010 (Electrical Engineering) \hfill \textbf{2010} \vspace{1mm}\\%
	%    IEEE Signal Processing Society Student Travel Grant to attend ICASSP held in Kyoto, 2012. \vspace{1mm}\\%
	PhD Scholarship, Ministry of Human Resource and Development, Govt. of India \hfill \textbf{2010-2016} \vspace{1mm}\\%
	Distinctive performance in National Science Olympiad\hfill \textbf{2000}                                      
		
	\section{\mysidestyle Extra-curricular activities}
			
			\begin{list2}
			
				\item
				
				
				Active Member of  Spicmacay, Voice club, IEEE and Management Forum
				at NITK Surathkal
				\item Active member of Mess committee at IISc Bangalore from July 2013 - May 2015
				\item
				
				Playing badminton,  running, swimming, biking, triathlon,  programming, technology, guitar,  photography, writing blogs
				
				\item Finished Ironman 70.3 in Hyderabad triathlon and Olympic triathlon in Goa triathlon events in 2017
				
				
		\end{list2}
	
				\section{\mysidestyle Languages known}
				English, Hindi and Telugu 
			%__________________________________________________________________________________________________________________
			%     \section{\mysidestyle Referees} 
			%     {\sl Available on request.}
			% %______________________________________________________________________________________________________________________
			
			% \vspace{-4.5mm}
			\section{\mysidestyle References} 
        \textsl{Available on request}  
%	\begin{tabular}{@{}p{6cm}p{6cm}}
				% \textbf{Professor Thippur V. Sreenivas}       &  \textbf{Dr Stephen So}                   \\
				% Professor, Dept. ECE                          &  Associate Lecturer                       \\
				% Indian Institute of Science             &  Griffith University                      \\
				% Banglore, India                         &  Gold Coast, Queensland, Australia        \\
				% % phone: \textsl{available on request}    &  phone: \textsl{available on request}     \\
				% % e-mail: \textsl{available on request}   &  e-mail: \textsl{available on request}    \\
				
			%	\vspace{6mm}
			%	\textbf{Vinodh Kumar Ravindranath}                         \\
			%	\vspace{.01mm}
			%	Chief Technology Officer                                             \\
			%	Belong.co                                     \\
			%%	Banglore, India                                                 \\
				% phone: \textsl{available on request}    &  phone: \textsl{available on request}     \\
			%	e-mail: \textsl{vinodhr@gmail.com}  \\
			%	Linkedin: \textsl{https://www.linkedin.com/in/vinodhkumarr/}\\
				
			%	\vspace{6mm}
			%	\textbf{Prof. A G Ramakrishnan}                         \\
			%	\vspace{.01mm}
			%%	Professor, Dept. EE                                            \\
		%		Indian Institute of Science                                     \\
		%		Banglore, India                                                 \\
				% phone: \textsl{available on request}    &  phone: \textsl{available on request}     \\
		%		e-mail: \textsl{agr@iisc.ac.in}  \\
				
				
		%		\vspace{6mm}
		%		\textbf{Dr. T V Ananthapadmanabha}                         \\
		%		\vspace{.01mm}
	%			Founder and Chief innovator,\\
		%		Voice and Speech Systems, Bangalore, India\\                                          
				% phone: \textsl{available on request}    &  phone: \textsl{available on request}     \\
		%		e-mail: \textsl{tva.blr@gmail.com}  \\
				
				
				
				
				%\vspace{6mm}
				%\textbf{Prof. Thippur V. Sreenivas}                         \\
				%Professor, Dept. ECE                                            \\
				%Indian Institute of Science                                     \\
				%Banglore, India                                                 \\
				% phone: \textsl{available on request}    &  phone: \textsl{available on request}     \\
				%e-mail: \textsl{tvsree@iisc.ac.in}  \\
				
				
				
				% \vspace{6mm}
				% \textbf{K.V.S. Hari}                         \\
				% Professor, Dept. ECE                                   \\
				% Indian Institute of Science                                     \\
				% Banglore, India                                                 \\
				% phone: \textsl{available on request}    &  phone: \textsl{available on request}     \\
				% e-mail: \textsl{available on request}   &  e-mail: \textsl{available on request}    \\
		%	\end{tabular}
			
			%______________________________________________________________________________________________________________________
		\end{resume}
		
	\end{document}
	
	
	%______________________________________________________________________________________________________________________
	% EOF
	
